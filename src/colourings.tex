\chapter{Colourings}
A proper colouring is a function $c: \v \rightarrow \text{set of colours}$ such that no two adjacent vertices in $G$ receive the same colour. The set of colours is generally a set of "real" colours or the set $[h] = \{1, \dots, k\}$ for some integer $k$. We say $G$ is $k$-colourable if there exists a proper colouring $c$ of $G$ with $\leq k$ colours. The minimum $k$ such that $G$ is $k$-colourable is called the chromatic number of $G$ and is denoted with $\mathcal{X}(G)$.

Note that $\cn[\clq{n}] = n$, in particular, $\clq{n} \subseteq G \implies \cn(G) \geq n$.
\begin{customtheorem}{6.1}
\label{theorem:6.1}
    Every planar graph is $4$-colourable.
\end{customtheorem}
If we want to colour vertices we have a very simple and greedy algorithm that works in the following way: Given an ordering $v_1, \dots, v_n$ of $\v$ we colour $G$ in this order at stage $i$ we colour $v_i$ with the lowest colour not used for any of its already coloured neighbours. Therefore $v_i$ is coloured with colour $\leq \deg[v_i] + 1$, which means that we use at most $\maxd + 1$ colours, thus $\cn \leq \maxd + 1$.

Remark: The colouring is not order-invariant, which means that the ordering off the vertices matters.

Intuition: To minimise the number of colours used, it seems sensible to put vertices of large degree at the start and small degree vertices at the end.

For a partial ordering $v_n, \dots, v_{i + 1}$ denote $G_i = G - \{v_n, \dots, v_{i + 1}\}$

Now chose an ordering as follows: Let $v_n$ be a vertex of degree $\mind$ in $G$ for $i = n - 1, \dots, 1$ in turn, choose a vertex $v_i$ of degree $\mind$ in $G_i$. Now apply the greedy algorithm with ordering $v_1, \dots, v_n$. Then, $\fa i \in [n]$, $v_i$ is coloured with a colour $\leq d[G_i]{v_i} + 1 \das \mind[G_i] + 1$. Therefore:
\begin{equation*}
    \cn \leq \max_{i \in [n]}\mind[G_i] + 1
\end{equation*}
\begin{customproposition}{6.2}
\label{proposition:6.2}
    For any graph $G$, $\cn \leq \max_{H \subseteq G} \mind[H] + 1$
\end{customproposition}
\begin{customcorollary}{6.3}
\label{corollary:6.3}
    Every graph $G$ has a subgraph of minimum degree $\geq \cn - 1$
\end{customcorollary}
\begin{customtheorem}{Headwood}
\label{theorem:headwood}
    Every planar graph is $5$-colourable.
\end{customtheorem}
\begin{prf}
    We prove the theorem by induction on $\nv$, the base case is trivial, we just have the one vertex.

    Let $G$ be a plane graph on at least two vertices. Without loss of generality we may add edges to $G$ so that it is maximally plane. So by Proposition \ref{lemma:4.8}, $G$ is a plane triangulation and has $3n - 6$ edges (by Corollary \ref{corollary:4.9}). By Handshaking lemma we know that
    \begin{equation*}
        \avgd = \frac{2 - (3n - 6)}{n} \leq 6
    \end{equation*}
    Therefore there exists a vertex $v \in \v$ of degree $\leq 5$.

    Apply the induction hypothesis to $G - v$ to obtain a $5$-colouring $c$ of $G - v$. If $c$ uses at most $4$ colours on $\ngbrs{v}$, then there is a free colour for $v$ and we can directly extend $C$ to a $5$-colouring of $G$. So, we may assume that $N(v) = \{v_1, \dots, v_5\}$ where $c(v_i) = i \fa i$.

    By Corollary \ref{corollary:4.19}, $G$ contains a cycle $\cycl$ with $\v[\cycl] = N(v)$. Without loss of generality we can say that $\cycl = v_1v_2\dots v_5$. Let $G_{1, 3}$ ne the subgraph of $G - v$ induced by the vertices coloured with $1$ or $3$.

    If $v_1$ and $v_3$ lie in distinct components $K^1$ and $K^3$ of $G_{1, 3}$, then we may swap colours $1$ and $3$ in $K^1$. This is still a colouring of $G - v$ but now $c(v_1) = 3$, so we can colour $v$ with $1$ and we may assume $v_1$ and $v_3$ are in a common component of $G_{1, 3}$, which implies that $G - v$ contains a $v_1 - v_3$ path $P_{1, 3}$ which only uses vertices coloured with $\{1, 3\}$.

    Similarly, we may assume that $G - v$ contains a $v_2 - v_5$ path $P_{2, 5}$ using vertices coloured with $\{2, 5\}$.

    So $G$ contains a subdivision for $\clq{5}$, with branch vertices $v, v_1, v_2, v_3, v_5$, but this contradicts \ref{theorem:kuratowski}'s theorem.\qed
\end{prf}
\begin{coolfact}
    A simple argument shows that any triangle-free planar graph has minimum degree at most $3$, so can show inductively that such graphs are $4$-colourable.
\end{coolfact}
\begin{customtheorem}{9.5}
\label{theorem:9.5}
    Every triangle free planar graph is $3$-colourable.
    \begin{equation*}
        \cn \leq \maxd + 1
    \end{equation*}
\end{customtheorem}
\begin{customtheorem}{Brooks}
\label{theorem:brooks}
    Let $G$ be a connected graph which is neither complete nor it is an odd cycle, then:
    \begin{equation*}
        \cn \leq \maxd
    \end{equation*}
\end{customtheorem}
\begin{prf}
    By induction on $\nv$:
    \begin{itemize}
        \item[\textbf(Base case)] If $\maxd \leq 2$, then $G$ is a path or a cycle, and so $G$ is $2$-colourable by assumption.
        \item[\textbf(Inductive case)] If we suppose that $\Delta \das \maxd \geq 3$.
        
        Suppose for a contradiction that $G$ is not $\Delta$-colourable and let $v \in \v$. Let $H = G - v$, if $H$ is neither complete nor an odd cycle, then by Induction Hypothesis $H$ is $\Delta$-colourable.

        If $H$ is complete or an odd cycle then it is $(\maxd[H] + 1)$-colourable but $H$ is $\maxd[H]$-regular, so as $G$ is connected, $v$ as a neighbour $w \in \v[H]$, which has degree $\maxd[H] + 1$ in $G$. Therefore $\Delta = \maxd[H] + 1$. Which means that, in all cases $H$ is $\Delta$-colourable.

        \begin{customclaim}{1}
        \label{claim:brooks_1}
            Each $\Delta$-colouring of $H$ uses all colours $1, \dots, \Delta$ on $\ngbrs{v}$.
        \end{customclaim}
        \begin{prf}
            If not then there would be a free colour that I can use, which means that $\cn > \Delta$, which is a contradiction.\qed
        \end{prf}
        Given a $\Delta$-colouring of $G$ we always denote by $v_1, \dots, v_\Delta$ the neighbours of $v$ with $v_i$ of colour $i$. And for colour $i$ and $j$ we denote by $H_{i, j}$ the subgraph of $G$ induced by the vertices of colours $i$ and $j$.
        \begin{customclaim}{2}
        \label{claim:brooks_2}
            For any $\Delta$-colouring of $H$ $v_i$ and $v_j$ lie in a common component of $H_{i, j}$.
        \end{customclaim}
        \begin{prf}
            If not, then we can swap colours $i$ and $j$ in the component containing $v_i$ so that now both $v_i$ and $v_j$ have colour $j$, a contradiction to Claim \ref{claim:brooks_1}.\qed
        \end{prf}
        \begin{customclaim}{3}
        \label{claim:brooks_3}
            For any $\Delta$-colouring of $H$ and $i, j \in [\Delta]$ the component $C_{i, j}$ of $H_{i, j}$ containing $v_i$ and $v_j$ is a $v_i - v_j$ path.
        \end{customclaim}
        \begin{prf}
            First, note that $\deg[H]{v_i} \leq \Delta - 1$ ($v_iv \in \e$) so if $v_i$ has $2$ neighbours of the same colour, then there is a colour $k \in [\Delta] \setminus \{i\}$ which is not used on $\ngbrs[H]{v_i}$ and so now $i$ is no longer used in $\ngbrs{v}$, a contradiction to Claim \ref{claim:brooks_1}. Since all neighbours of $v_i$ in $C_{i, j}$ are coloured $j$, this implies that $\deg[C_{i, j}]{v_i} = 1$, similarly $\deg[C_{i, j}]{v_j} = 1$. Let $P$ be a $v_i - v_j$ path in $H_{i, j}$ and suppose for a contradiction that $P \neq C_{i, j}$. Since $v_i$ and $v_j$ have degree $1$ in $C_{i, j}$ this means that $P$ has an internal vertex of degree $3$ in $C_{i, j}$. Let $u$ be the first one in $P$. And note that all $3$ neighbours of $u$ in $C_{i, j}$ have the same colour. So, since $\deg{u} \leq \Delta$, there is a colour $h \in [\Delta] \setminus \{i, j\}$ which is not used on $\ngbrs[H]{u}$. Recolouring $u$ with $h$ gives a colouring where $v_i$ and $v_j$ lie in distinct components of $H_{i, j}$ a contradiction to Claim \ref{claim:brooks_2}.\qed
        \end{prf}
        \begin{customclaim}{4}
        \label{claim:brooks_4}
            For any $\Delta$-colouring of $H$ and any distinct $i, j, k \in [\Delta]$, $C_{i, j}$ and $C_{j, k}$ only intersect in $v_i$.
        \end{customclaim}
        \begin{prf}
            Suppose for a contradiction that $\ex u \neq v_i$ with $u \in \v[C_{i, j}] \cap \v[C_{j, k}]$.

            Note $v$ has to be coloured $i$ and so is an internal vertex of $C_{i, j}$ and $C_{i, k} \implies u$ has two neighbours coloured $j$ and two neighbours coloured $k$. So $\ngbrs{u}$ ised $\leq \Delta - 2$ colours, which means, in turn, that there is $L \in [\Delta] \setminus \{i\}$ such that no neighbour of $u$ is coloured $L$. So we could recolour $u$ with $L$ but then $v_i$ and $v_j$ would lie in distinct components of $H_{i, j}$, a contradiction to Claim \ref{claim:brooks_2}. \qed
        \end{prf}
        Suppose $\ngbrs{v}$ is complete. Then $G[v \cup \ngbrs{v}] \cong \clq{\Delta + 1}$. So, since $\maxd = \Delta$ and $G$ is connected, we have $G = G[v \cup \ngbrs{v}] \cong \clq{\Delta + 1}$. Which is a contradiction.

        So we may assume withour loss of generality that $v_1v_2 \notin \e$ in some colouring $C$. By Claim \ref{claim:brooks_3}, $C_{1, 2}$ is a path and by assumption, of length at least $2$. So let neighbours of $v_1$ on $C_{1, 2}$. note $C(v) = 2$.

        Recall that $\Delta \geq 3$, so again, by Claim \ref{claim:brooks_3} implies that $C_{1, 3}$ is a $v_1 - v_3$ path which only intersects $C_{1, 2}$ at $v_1$ (by Claim \ref{claim:brooks_4}).

        Now define a new $\Delta$-colouring $C'$ of $H$ by swapping the colours $1$ and $3$ along $C_{1, 3}$. This means that $v_1' = v_3$ and $v_2' = v_2$.

        Applying Claim \ref{claim:brooks_3}, $C_{1, 2}$ is a $v_1' - v_2$ path and since $uC_{1, 2}v_2$ was not recoloured, we get that $uC_{1, 2}v_2 \subseteq C'_{1, 2}$. So $u$ is an internal vertex of $C'_{1, 2}$ which means that $u$ has $2$ neighbours of colour $1$ on $C'_{1, 2}$ and these are not $v_1$ and also not on $C_{1, 3}$ so $u$ actually has $3$ neighbours in $C_{1, 2}$. This contradicts the fact that $C_{1, 2}$ is a path (by Claim \ref{claim:brooks_3}). \qed
    \end{itemize}
\end{prf}
\noindent The clique number $w(G)$ is the largest $k$ such that $G$ contains a $\clq{k}$ as a subgraph. As an example, if $w(G) = 2$ then $G$ is a bipartite graph.

\noindent\textbf{Conjecture (Barodin - Kostochka)} - Let $G$ be a graph with $\maxd \geq 9$. Then $\cn \leq \maxd - 1$ unless $w(G) \geq \maxd$.

Why $\maxd \geq 9$? Because there is a graph $G$ with $\maxd = 8, w(G) < 8 \land \cn \geq 8$.

\begin{customlemma}{6.8}
\label{lemma:6.8}
    For any graph $G$,
    \begin{equation*}
        \cn \geq \frac{\nv}{\alpha(G)}
    \end{equation*}
\end{customlemma}
\begin{prf}
    Fix an optimal colouring of $G$ (a $\cn$ colouring).

    Let $C_1, \dots, C_{\cn}$ be the colour classes (which means $C_i$ is the set of vertices coloured $i$). Note that each $C_i$ induces an independent set so $|C_i| \leq \alpha(G) \fa i$ therefore:
    \begin{equation*}
        \nv = \sum_{i \in [\cn]}|C_i| \leq \cn \cdot \alpha(G)
    \end{equation*}
    And this proves the theorem.\qed
\end{prf}

\begin{customlemma}{6.9}
\label{lemma:6.9}
    For any $G$, we have $\cn \geq w(G)$. On the other hand, there is no function $f$ such that $\cn \leq f(w(G))$ for all $G$.
\end{customlemma}
In fact, there are triangle-free graphs $(w(G) = 2)$ with arbitrarily large chromatic number.
\begin{customtheorem}{Mycielski}
\label{theorem:mycielski}
    For any positive integer $k$, there exists a triangle-free graph with chromatic number $k$.
\end{customtheorem}
\begin{customlemma}{6.11}
\label{lemma:6.11}
    For any optimal colouring of $G$ and any $i \in [\cn]$, there exists $v_i \in \v$ coloured $i$ such that all colours appear in $\{v_i\} \cap \ngbrs{v_i}$.
\end{customlemma}
\begin{prf}
    Suppose for a contradiction that there is $i \in [\cn]$ such that for all $v \in \v$ coloured $i$ there is $j_v \in [\cn] \setminus \{i\}$ which is not on $\ngbrs{v}$. now recolour each $v$ of colour $i$ with colour $j_v$ which means that we have a $(\cn - 1)$ colouring. Which is contradictiory.\qed
\end{prf}
We can now prove \ref{theorem:mycielski}'s theorem.
\begin{prf}
    Induction on $k$.

    $G_1 =$ one certex.

    $G_2 =$ bipartite graph.
    \begin{itemize}
        \item[\textbf(Base case)] Let $G_3$ be an odd cycle of length $\geq 5$. Clearly triangle-free and $\cn[G_3] = 3$.
        \item[\textbf(Inductive case)] Suppose inductively that we have constructed $G_k$ such that it's triangle-free with $\cn[G_k] = k$.
        
        We define $G_{k + 1}$ as follows: Let $V_k = \v[G_k]$ and let $W_k$ consist of one copy $v'$ of each $v \in V_k$. Let $z$ be a new vertex $\notin V_k \cup W_k$.

        Let $\v[G_{k + 1}] = V_k \cup W_k \cup \{z\}$

        Now let $G_{k + 1}[V_k] \cong G_k$ and let $N(z) \das W(k)$ and for each $v' \in W_k$, let $N(v') \das N(v)$.

        $G_{k + 1}$ is triangle free:
        \begin{itemize}
            \item $W_k$ is an independent set of $z$ is not in a triangle and a triangle contains at most $1$ vertex from $W_k$.
            \item By the induction hypothesis we have no triangle in $G_{k + 1}[V_k]$
            \item So the only remaining possibility is a triangle $w'uv$ for $w' \in W_k$ and $u, v \in V_k$. But $w'uv$ is a triangle, which means that $wuv$ is also a triangle in $G_k$, which is contradicting the inductive hypothesis.
        \end{itemize}
        $\cn[G_{k + 1} \leq k + 1]$: Construct a $(k + 1)$-colouring $c$ of $G_{k + 1}$ as follows
        \begin{itemize}
            \item By induction hypothesis it's possible to colour $V_k$ with $k$ colours.
            \item Then for each $v \in W_k$, colour $v'$ as $v$.
            \item Colour $z$ with colour $k + 1$.
        \end{itemize}
        $\cn[G_{k + 1} \leq k + 1]$: Suppose not, and let $\emptyset$ be a $k$-colouring of $G_k$ and so, by Lemma \ref{lemma:6.11} gives that $\fa i \in [k]$, there exists $v_i \in V_k \st \emptyset(v_i) = i$ and $\emptyset(\{v_i\} \cup (N(v_i) \cap V_k)) = [k]$. But then $N(v_i') \cap V_k = N(v_i) \cap V_k$ so $\emptyset(v_i') = i$.

        Therefore $W_k = N(z)$ has a vertex of each colour. So $z$ has a neighbour of colour $\emptyset(z)$, which is a contradiction.\qed
    \end{itemize}
\end{prf}
\section{Edge Colourings}
An edge colouring of $G$ is a function from $\e$ to a set of colours such that no two adjacent edges have the same colour. We say $G$ is $k$-edge-colourable if it can be edge-coloured with $k$ colours.

The chromatic index $\ci$ is the minimum $k$ such that $G$ is $k$-edge-colourable. Let's now go through a couple of observations:
\begin{itemize}
    \item $\ci \geq \maxd$
    \item An edge colouring of $G$ is equivalent to a vertex colouring of $L(G)$, which is the line graph.
    \item The colour classes in the vertex colourings induce independent sets, while edge colourings induce matchings.
\end{itemize}
Therefore the graph induced by two colour classes is always a union of disjoint paths and cycles.
\begin{customtheorem}{6.15}
\label{theorem:6.15_konig}
    For any bipartite graph $G_i, \ci = \maxd$    
\end{customtheorem}
\begin{customproposition}{6.16}
\label{proposition:6.16}
    For any graph $G$ there exists a $\maxd$-regular graph which contains $G$ as a subgraph. Moreover, if $G$ is bipartite then we can take $G'$ to be bipartite as well.
\end{customproposition}
We can now prove Theorem \ref{theorem:6.15_konig}, or König's theorem.
\begin{prf}
    Let $G$ be a bipartite graph. by proposition \ref{proposition:6.16} we may assume without loss of generality that $G$ is regular, since we can find a supergraph containing $G$ and $\maxd$-regular and bipartite, therefore any $\maxd$-edge-colouring for $G'$ is a $\maxd$-edge-colouring in $G'$.

    By Corollary \ref{corollary:2.4} $G$ has a perfect matching $M$, inductively we may assume $G - M$ is $(\maxd - 1)$-edge-colourable therefore we can extend the edge colouring by adding one colour and colouring every edge in $M$ the same.\qed
\end{prf}
\begin{customtheorem}{Vizing}
\label{theorem:vizing}
    For any graph $G, \maxd \leq \ci \leq \maxd + 1$
\end{customtheorem}
Note that because of the above theorem we can divide edge-colourable graphs into two classes:
\begin{itemize}
    \item Class 1 - $\ci = \maxd$
    \item Class 2 - $\ci = \maxd + 1$
\end{itemize}
We can now move to the proof for the theorem.
\begin{prf}
    Fix $\Delta$. We will show that, for any graph $G$, with $\maxd \leq \Delta$, we have $\ci \leq \Delta + 1$.

    In this proof, whenever we say "edge colouring" we mean an edge colouring with colours $1, \dots, \Delta + 1$. We will reason by induction on $\ne$.
    \begin{itemize}
        \item[\textbf(Base case)] If there are no edges then we have a colouring with $0$ colours.
        \item[\textbf(Inductive case)] Suppose that $\ne \geq 1$ Let $xy_0 \in \e$ by induction we know that $G - xy_0$ is $(\Delta + 1)$-edge-colourable.
        
        We can guarantee that, since $x$ is using $\Delta - 1$ colours and $y_0$ is using $\Delta - 1$ colours as well, this means that I have at most $2$ free colours. We say that the colour $\alpha$ is missing from vertex $x$.

        If $\alpha$ is missinga at a generic vertex $v$ and $\beta$ is another colour, then the component of the graph induced by colours $\alpha$ and $\beta$ which contains $v$ is a path with $v$ as an endpoint (potentially of length $0$ if $\beta$ is also missing at $v$). This path will be called the $\alpha,\beta$-path from $v$.

        Fix an edge colouring of $G - xy_0$. now let $y_0y_1\dots y_k$ be a maximal sequence of distinct neighbours of $x$ such that for each $i \in [k], c(xy_i)$ is missing at $y_{i - 1}$. Let $\alpha$ ne a colour missing at $y_k$. Let $P$ be the $\alpha, \beta$-path from $y_k$.
        \begin{itemize}
            \item[Case 1] \textbf{$\bm{P}$ does not end at $\bm{x}$}
            
            Note that $x \notin \v[P]$ because $\alpha$ is missing at $x$. We can swap colours $\alpha$ and $\beta$ along $P \implies y_k$ was missing $\beta$ but now $y_k$ is missing $\alpha$.

            For each $i \in k$ we recolour the edge $xy_{i - 1}$ with colour $c(xy_1)$ and this is possible since, by assumption, $c(xy_i)$ is missing at $y_{i - 1}$.
            \item[Case 2] \textbf{$\bm{P}$ ends at $\bm{x}$}
            
            Say $y = y_j$ swap the colour $\alpha$ and $\beta$ along $P$. Note $xy_j$ is now coloured $\alpha$ and $\beta$ is no longer used at $x$. For all $i \in [j]$ recolour $xy_{i - 1}$ with $c(xy_i)$.\qed
        \end{itemize}
    \end{itemize}
\end{prf}