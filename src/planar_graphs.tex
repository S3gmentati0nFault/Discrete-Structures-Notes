\chapter{Planar graphs}
\begin{definition}[Planar graph]
    A graph that can be drawn on the plane so that no two edges cross.
\end{definition}
Such an embedding on the plane is called a planar emberdding or planar drawing, A plane graph can be defined as a planar graph with a mapping from every node to a point on a plane, and from every edge to a plane curve on that plane, such that the extreme points of each curve are the points mapped from its end nodes, and all curves are disjoint except on their extreme points\footnote{\url{https://en.wikipedia.org/wiki/Planar_graph}}.

An example of a planar graph is $\clq{4}$.
% TODO: ADD GRAPH FOR K_4

\begin{customtheorem}{Euler}
\label{theorem:euler}
    A connected plane graph on $n$ vertices with $m$ edges and $f$ faces satisfies the following equation:
    \begin{equation*}
        n  - m + f = 2
    \end{equation*}
\end{customtheorem}
A plane graph $G$ is maximally plane if, for any non-adjacent vertices $x, y \in \v$, any drawing of the edge $xy$ would induce a crossing. An example is the following:
% TODO: ADD EXAMPLE OF A MAXIMALLY PLANE PLANAR GRAPH
\begin{definition}[Plane triangulation]
    A plane triangulation is a plane graph whose faces are all bounded by a triangle.
\end{definition}
\begin{customlemma}{4.8}
\label{lemma:4.8}
    A plane graph on $\leq 3$ vertices is maximally plane $\iff$ it is a plane triangulation
\end{customlemma}
\begin{prf}
    \begin{itemize}
        To prove the above lemma we will prove both directions
        \item [($\impliedby$)] Suppose $G$ is a plane triangulation. The only edges that could be added without crossing are edges fully contained within a face of $G$ joining two vertices from the boundary of $f$. But every boundary is a triangle, which is a complete graph.
        \item [($\implies$)] Suppose $G$ is maximally plane.
        \begin{customclaim}{1}
        \label{claim:4.8_1}
            $G$ is $2$-connected.
        \end{customclaim}
        \begin{prf}
            Suppose not, then we have either a cut vertex or a series of connected components, then $G = G_1 \cup G_2$, where $|\v[G_1] \cap \v[G_2]| \leq 1$ and $\v[G_1] \setminus \v[G_2] \neq \emptyset \neq \v[G_2] \setminus \v[G_1]$.

            Let $x \in \v[G_1] \setminus \v[G_2]$ and $y \in \v[G_2] \setminus \v[G_1]$ both lying on the boundary of the outside face. Then we can draw the edge xy within the outside face, without crossings, but that contradicts the plane maximality, thus the graph is $2$-connected.\footnote{
                In the proof we use a connection between two vertices that are on the border of the graph, the edge is positioned on the outside face, but nothing stops us from having one of the two graphs inside the other, nothing changes in the proving process though.
            }\qed
        \end{prf}
        By Lemma \ref{lemma:4.6} all the faces of $G$ are bounded by a cycle. Suppose for a contradiction that $G$ has a face which is bounded by a cycle $\cycl$ of length $\geq 4$.

        Let $x, y, z, w \in \v[\cycl]$ be distinct vertices in this order on $\cycl$, we could draw an edge $xz$ fully into $f$ without any crossing, so the fact that $G$ is maximally plane implies that $xz$ is already an edge in $G$, drawn outside of $f$.

        Similarly $yw \in \e$ and it is drawn otuside of $f$, but then $xz$ and $yw$ must cross, and that is a contradiction.\qed
    \end{itemize}
\end{prf}
\begin{customcorollary}{4.9}
\label{corollary:4.9}
    Any plane triangulation graph on $n$ vertices has $3n - 6$ edges and any plane graph has $\leq 3n - 6$ edges.
\end{customcorollary}
\begin{prf}
    Let $G$ be a plane triangulation, double count the number $N$ of pairs $(e, f)$ where $f$ is a face of $G$ and $e$ is an edge on the boundary of $f$. Since plane triangulation means that every face has $3$ edges on its boundary then:
    \begin{equation*}
        N = 3 \cdot \text{\# faces of $G$}
    \end{equation*}
    % TODO: CHECK THE NUMBER OF THE LEMMA BELOW
    every edge is on the boundary of $2$ faces (By Lemma \ref{lemma:4.4}) thus:
    \begin{equation*}
        2 \cdot \ne = N = \text{\# faces of $G$}
    \end{equation*}
    If we call $F(G)$ the set of faces of graph $G$ then:
    \begin{equation*}
        3 \cdot |F(G)| = 2 \cdot \ne
    \end{equation*}
    But by \ref{theorem:euler}'s formula:
    \begin{align*}
        &\nv - \ne + \frac{2}{3}\ne = 2 \\
        &- \frac{1}{3}\ne = -\nv + 2 \\
        &\ne = 3 \cdot \nv - 6
    \end{align*}
    For a general plane graph $G$ we add edges until we get a maximally plane graph $G'$. By Lemma \ref{lemma:4.8} and the above, $\ne[G'] = 3 \cdot n - 6$. Since $G \subseteq G' \implies \ne \leq \ne[G'] = 3 \cdot n - 6$.\qed
\end{prf}
\begin{definition}[Plane Quadrangulation]
    A plane quadrangulation is a plane graph whose faces are all bounded by a cycle of length 4.
\end{definition}
\begin{customcorollary}{4.10}
\label{corollary:4.10}
    Any plane quadrangulation on $n$ vertices has $2 \cdot n - 4$ edges and any plane bipartite graph on $n \geq 4$ vertices has $\leq 2 \cdot n - 4$ edges.
\end{customcorollary}
\begin{prf}
    Let $G$ be a plain graph on $n$ vertices. Double count the number $N$ of pairs $(e, f)$ where $f$ is a face and $e$ is an edge on the boundary. If $G$ is a plane quadrangulation:
    \begin{equation*}
        2 \cdot \ne = N = 4 \cdot \text{\# of faces}
    \end{equation*}
    So \ref{theorem:euler}'s formula gives $n - m + f = 2$ thus I get:
    \begin{equation*}
        m = 2 \cdot n - 4
    \end{equation*}
    Now suppose the graph is bipartite, we keep on adding edges as long as $G$ is still bipartite and plane. Let $G'$ be the resulting graph, note $G'$ is connected and $2$-connected (similar arguments as before). So all faces of $G$ are bounded by a cycle (By Lemma \ref{lemma:4.6}) of length $\geq 4$ to make it bipartite:
    \begin{equation*}
        2 \cdot \ne[G'] \geq N \geq 4 \cdot \text{\# faces}
    \end{equation*}
    So by \ref{theorem:euler}'s formula, again we have:
    \begin{align*}
        &n - \ne[G'] + \frac{\ne[G']}{2} \geq 2 \\
        &\ne[G'] \leq 2 \cdot n - 4
    \end{align*}
    And that proves the corollary.\qed
\end{prf}
\begin{customcorollary}{4.11}
\label{corollary:4.11}
    $\clq{5}$ and $\clq{3, 3}$ are not planar.
\end{customcorollary}
\begin{definition}[Platonic solid]
    A platonic solid is a $3$-dimensional convex polyhedra such that every vertex meets all the same number of faces, and each face is a regular polygon.
\end{definition}
An example of a platonic solid is a cube.
\begin{customtheorem}{4.12}
\label{theorem:4.12}
    There are only $5$ platonic solids
\end{customtheorem}
\begin{prf}
    Note that any platonic solid is embeddable on the sphere, and so can be projected onto the plane, to obtain a plane graph with all faces bounded by a cycle of length $r$ (for some $r \geq 3$) and every vertex has a degree $s$ (for some $s \geq 3$). That implies that $r \cdot f = 2 \cdot m$, which means that $r$ edges are on the boundary between two faces. By \ref{lemma:handshaking} $2 \cdot m = n \cdot s$.By \ref{theorem:euler}'s formula:
    \begin{align*}
        &\frac{2m}{s} - m + \frac{2m}{r} = 2 \\
        &m \cdot \tnd{\frac{1}{s} + \frac{1}{r} - \frac{1}{2}} = 1 \text{\hspace{1cm} What's in the parentheses has to be $< 1$} \\
        &\frac{1}{s} + \frac{1}{r} = \frac{1}{m} + \frac{1}{2} \\
        &\frac{1}{s} + \frac{1}{r} > \frac{1}{2}
    \end{align*}
    Together $s, r \geq 3$, we get $s, r \leq 5$ and if one of $s, r$ is $3$ then the other is $5$. We have the following $5$ solutions for $(r, s)$:
    \begin{itemize}
        \item $(3, 3)$ is a tetrahedron
        \item $(3, 4)$ is an octahedron
        \item $(3, 5)$ is an icosahedron
        \item $(4, 3)$ is a cube
        \item $(5, 3)$ is a dodecahedron
    \end{itemize}
    Since the possibilities are just these $5$ this proves our theorem.\qed
\end{prf}
\section*{Kuratowski's theorem}
\begin{customtheorem}{Kuratowski}
\label{theorem:kuratowski}
    The following are equivalent:
    \begin{enumerate}
        \item $G$ is planar.
        \item $G$ has neither $\clq{5}$ nor $\clq{3, 3}$ as a minor.
        \item $G$ has neither $\clq{5}$ nor $\clq{3, 3}$ as a topological minor.
    \end{enumerate}
\end{customtheorem}
Let's recall that if a graph $G$ has a topological minor $X$ then $X$ is also minor for $G$ but the opposite is not trivially true. Let's also recall that a subdivision is a graph obtained by replacing some edges by internally disjoint paths. The vertices of G are called the branch vertices of the subdivision and the new vertices are called subdivding vertices.
\begin{customlemma}{4.14}
\label{lemma:4.14}
    No planar graph has $\clq{5}$ or $\clq{3, 3}$ as a topological minor.
\end{customlemma}
\begin{prf}
    If $G$ is planar and has $\clq{5}$ or $\clq{3, 3}$ as a topological minor, then this induces a planar embedding of a subdivision of $\clq{5}$ or $\clq{3, 3}$ so we could obtain a planar embedding of $\clq{5}$ or $\clq{3, 3}$ simply by "erasing" the subdividing vertices and replacing the subdivided paths with an edge.\qed
\end{prf}
\begin{customlemma}{4.15}
\label{lemma:4.15}
    A graph contains $\clq{5}$ or $\clq{4, 4}$ as a minor $\iff$ it contains $\clq{5}$ or $\clq{3, 3}$ as a topological minor.
\end{customlemma}
\begin{prf}
    To prove this lemma we will prove both directions separately
    \begin{itemize}
        \item [($\implies$)] Clear by definintion (all topological minors are minors)
        \item [($\impliedby$)] Let $G$ be a graph with $\clq{5}$ or $\clq{3, 3}$ as a minor, we need to find $\clq{5}$ or $\clq{3, 3}$ as a topological minor. By Proposition \ref{proposition:1.27}, if $\clq{3, 3} \minr G$, then $G$ contains a subdivision of $\clq{3, 3}$.
        
        Suppose $\clq{5} \minr G$. Let $H$ be a minimal subgraph of $G$ such that $\clq{5} \minr H$. For $x \in \v[\clq{5}]$, let $V_x$ denote the corresponding branch set in $H$.

        By minimality we have the following properties:
        \begin{itemize}
            \item $H[V_x]$ is a tree $\fa x \in \v[\clq{5}]$.
            \item There is a single edge between $V_x$ and $V_y, \fa xy \in \e[\clq{5}]$.
            \item Each leaf of $H[V_x], x \in \v[\clq{5}]$ sends an edge to another $V_y, y \in \v[\clq{5}]$.
        \end{itemize}
        For each $x \in \v[\clq{5}]$, let $T_x \subseteq H$ be obtained from $H[V_x]$ by adding all edges with an endpoint in $V_x$, note that $T_x$ is a tree with $4$ leaves. if each $T_x$ contains a vertex of degree $4$m then these are branch vertices for a $\clq{5}$ subdivision.

        % TODO: UNDERSTAND AND FIX THIS PROOF, FROM NOW ON IT MAKES LITTLE TO NO SENSE.
        Suppose there is $T_x$ with no vertex of degree $4$, since only $4$ leaves, $\maxd[T_x] \leq 3$. Since in the case of $\maxd[T_x] = 1$ $X$ is an edge and in the case of $\maxd[T_x] = 2$ $X$ is a path (thus we have two leaves). So $\maxd[T_x] = 3$  and $T_x$ has exactly two vertices $u, v$ of degree $3$.

        By Proposition \ref{proposition:1.27}, it is enough to find $\clq{3, 3}$ as a minor:
        \begin{itemize}
            \item Contract $V_x$ to a single edge between $u$ and $v$,
            \item Contract each other $V_y$ to a single vertex, now we have (after deleting some edges) $\clq{3, 3}$ as a subgraph with bipartition $(\grf{u, v, v'}, \grf{v, w, w'})$ where $v'$ and $v"$ are neighbours of $v$ thus $\clq{3, 3} \minr H \implies \clq{3, 3}$ as a subdivision.\qed
        \end{itemize}
    \end{itemize}
\end{prf}
\begin{customlemma}{4.16}
\label{lemma:4.16}
    Every $3$-connected graph without $\clq{5}$ or $\clq{3, 3}$ as a minor is planar.
\end{customlemma}
\begin{prf}
    We prove the statement by induction on $n := \nv$. Let $G$ be $3$-connectew with $\clq{5}, \clq{3, 3} \nminr G$.

    \textbf{Base case: } $n = 4 \implies G$ is $\clq{4}$, which is a planar clique. 
    
    \textbf{Induction hypothesis: }Suppose $n > 5$. By \ref{lemma:3.4}, there is $xy \in \e$ such that $G / xy$ is still $3$-connected. 
    
    \textbf{Inductive case: }Note $G / xy$ still does not contain $\clq{5}$ or $\clq{3, 3}$ as a minor, so by induction hypothesis $G / xy$ is planar. Let $V_xy$ be the new vertex in $G / xy$ and consider $G /xy$ as a plane graph (which means that we fix and work with an embedding). Let $f$ be the face of $G / xy - V_xy$ containing $V_xy$.

    Note $G / xy$ is $3$-connected, so $G / xy - V_xy$ is 3 connected, thus by Lemma $f$ is bounded by a cycle $\cycl$. Note $\ngbrs{x} \cup \ngbrs{y} \setminus \grf{x, y} = \ngbrs[G / xy]{V_xy} \subseteq \v[\cycl]$.

    If $|\ngbrs{x} \cap \ngbrs{y}| \geq 3$ then $G$ contains $\clq{5}$ as a minor, thus:
    \begin{itemize}
        \item Delete all vertices outside $\v[\cycl] \cup \grf{x, y}$,
        \item Contract $\cycl$ to a triangle on $3$ common neighbours of $x$ and $y$.
    \end{itemize}
    If $\cycl$ contains $a, b, c, d$ in this cyclic order with $a, c \in \ngbrs{x}$ and $b, d \in \ngbrs{y}$, then $G$ contains $\clq{3, 3}$ as a minor. Thus:
    \begin{itemize}
        \item Delete all vertices outside $\v[\cycl] \cup \grf{x, y}$,
        \item Contract $\cycl$ to cycle $abcd$,
        \item Take $\tnd{\grf{x, b, d}, \grf{y, a, c}}$ as a bipartition.
    \end{itemize}

    So there exists a subgraph $P \subseteq \cycl$ such that $\ngbrs{y} \setminus \grf{x} \subseteq \v[P]$ and $\ngbrs{x}$ intersects $\v[P]$ only at its endpoints. So draw $x$ in $f$ and connect it to all its neighbours on $\cycl$ without crossings. Note $xuPvx$ is the boundary $\cycl'$ of a face $f$ and with $\ngbrs{u} \subseteq \v[\cycl'] \implies$ can embed $y$ in $f'$. 
\end{prf}
\begin{customlemma}{4.17}
\label{lemma:4.17}
    Let $\mathcal{X}$ be a set of $3$-connected graphs. Let $G$ be a graph, and let $G_1, G_2$ be proper induced subgraphs such that $G = G_1 \cup G_2$ and such that $V(G_1 \cap G_2)$ is a smallest separator of size $\leq 2$ (thus it's not a $3$-connected graph). If $G$ is edge-maximal without a topological minor in $\mathcal{X}$, then $G_1$ and $G_2$ are also edge-maximal and $G_1 \cap G_2 = \clq{2}$.
\end{customlemma}
\begin{prf}
    Edge maximality here implies that for any $e \notin \e, G + e$ contains a subdivision of a graph in $\mathcal{X}$ which covers $e$.
    \begin{itemize}
        \item \boldmath$|\v[G_1] \cap \v[G_2]| = 0$\unboldmath
        
        Then, let $v_1 \in \v[G_1]$ and $v_2 \in \v[G_2]$. Edge maximality implies that $G + e$ contains a subdivision $S_0$ of $X_0 \in \mathcal{X}$ such that $v_1, v_2 \in \e[S_0]$, thus $X_0$ is not $3$-connected, since $v_1$ separates $G_1$ from $G_2$.
        \item \boldmath$|\v[G_1] \cap \v[G_2] = 1|$\unboldmath 
        
        Then, let $v$ be our separator, $v_1 \in \ngbrs{v} \cap \v[G_1]$ and $v_2 \in \ngbrs{v} \cap \v[G_2]$. ($v_1$ and $v_2$ exist since $G_1$ and $G_2$ are proper, that means that $\v[G_1] \setminus \v[G_2] \neq \emptyset$ and $\v[G_2] \setminus \v[G_1] \neq \emptyset$ and by the above $G$ is connected).
        
        Since $\grf{v}$ is a separator, $v_1v_2 \in \e$ so edge by edge-maximality $G + v_1v_2$ contains a subdivision $S_1$ of some $X_1 \in \mathcal{X}$ with $v_1v_2 \in \e[S_1]$.

        Note $\grf{v, v_1}$ separates $G_1$ and $G_2$ in $G + v_1v_2$ so since $X_1$ is $3$-connected, so tehe branch vertices (By \ref{theorem:menger}'s theorem we have $3$ vertex disjoint paths between any two branch vertices) of $S_1$ are in $G_2$ or $G_2$ (we can say this without loss of generality) and $\v{G_2}$ in $\v{S_1}$ is fully contained into a path $P \subseteq S$, between two branch vertices. NOte that $e \in \e[P]$ and $v \in \v[P]$ so we can obtain a new subdivision of $X_1$ by replacing $P$ in $S_1$ by $Pv_1vP$, but $S_1' \subseteq G$.
        \item \boldmath$\v[G_1] \cap \v[G_2] = \grf{u, v}$\unboldmath
        
        Suppose for a contradiction that $uv \notin \e$, by edge maximaplity $G + uv$ contains a subdivision $S_2$ of some $X_2 \in \mathcal{X}$ such that $uv \in \e[S_2]$. Note $\grf{u, v}$ is still a separator in $G + uv$, so, as before, all branch vertices of $S_1$ are, without loss of generality, in $G_1$ (by \ref{theorem:menger}'s theorem).

        Since $e \in \e[S_1]$, not that $S_1$ does not intersect $G_2 - G_1$ (no part of subdivision if going into $G_2$). Since $G$ is $2$-connected ($\grf{u, v}$ is a smallest separator), thus there is a $u-v$ path $Q$ in $G_2$ and we obtain a new subdivision $S_2'$ of $X_2$ by replacing $e$ with $Q$ in $S_2$. Thus $S_2' \subseteq G$, which is absurd because we said that $G$ has no subdivision.
    \end{itemize}
    It remains to show that $G_1$ and $G_2$ are also edge-maximal without topological minor in $\mathcal{X}$.

    Note any subgraph of $G_1$ or $G_2$ is also a subgraph of $G$. So neigher $G_1$ nor $G_2$ contains a subdivision of a graph in $\mathcal{X}$. So we only need to show that adding an edge in $G_1$ or $G_2$ creates a topological minor of some $x \in \mathcal{X}$ in $G_1$ or $G_2$ respectively. The same can be proven for $G_1$ by symmetry.

    Se let $v_1, v_2 \in \v[G_1]$ be non-adjacent, we want a subdivision from $\mathcal{X}$ in $G_1 + v_1v_2$. We know already that $G_1 + v_1v_2$ contains a subdivision $S$ of some $x \in \mathcal{X}$ with $e \in \e[S]$. Since $X$ is $3$-connected, by \ref{theorem:menger}'s theorem we know that all branch vertices are in $G_1$ or $G_2$.
    \begin{itemize}
        \item[Case 1: ] All branch vertices are in $G_2$. If $\v[S] \cap \v[G_2 - G_1] = \emptyset$, then we are done. Otherwise $\v[S] \cap \v[G_2 - G_1]$ is contained into a path $P$ between two branch vertices and goes through $u$ and $v$. (This is because $\grf{u, v}$ separates $G_1$ and $G_2$). Then obtain a subdivision $S'$ of $X$ in $G_1 + e$ by replacing $P$ with $PuvP$.
        \item[Case 2: ] All branch vertices are in $G_2$. Same arguments given above, it yields a subdivision of $X$ in $G_2$  
    \end{itemize}
    And this proves the theorem.\qed
\end{prf}
\begin{customlemma}{4.18}
\label{lemma:4.18}
    Every graph $G$ on $\geq 4$ vertices that is edge-maximal without $\clq{5}$ or $\clq{3, 3}$ as a topological minor is $3$-connected.
\end{customlemma}
\begin{prf}
    \textbf{Proof for \ref{theorem:kuratowski}'s theorem}.

    \begin{itemize}
        \item[($(1.) \implies (3.)$)] Proven by Lemma \ref{lemma:4.14}
        \item[($(2.) \iff (3.)$)] proven by Lemma \ref{lemma:4.15}
        \item[($(3.) \iff (1.)$)] Let $G$ without $\clq{5}$ or $\clq{3, 3}$ as a topological minor. Add edges to $G$ to obtain the supergraph $G' \supseteq G$ which is edge-maximal with no $\clq{5}$ or $\clq{3, 3}$ as a topological minor.
        
        By Lemma \ref{lemma:4.18}, $G'$ is $3$-connected. By Lemma \ref{lemma:4.15}, $G'$ has no $\clq{5}$ or $\clq{3, 3}$ as a minor, so $G'$ is planar by Lemma \ref{lemma:4.16}. So $G$ is also planar (since the bigger graph is planar then the subgraph must be planar as well).\qed
    \end{itemize}
\end{prf}