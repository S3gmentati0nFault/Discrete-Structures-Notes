\chapter{The basics}
\begin{definition}[Graph]
    A graph is a pair $(V, E)$ of finite disjoint sets where:
    \begin{itemize}
        \item $V$ is the set of vertices
        \item $E$ is a set of pairs, $E \subseteq \binom{V}{2}\footnote{The binomial coefficient has an interpretation in the field of combinatorics: it gives the number of ways, disregarding order, that $k$ objects can be chosen among $n$ objects. More formally: it defines the number of $k$-element subsets of an $n$-element set.} = \{(u, v): u \neq v, \text{with } u, v \in V\}.$
    \end{itemize}
\end{definition}
As a default, for the rest of the course, we will be considering simple graphs, which means that between any two $u, v$ vertices, there cannot be more than one edge. Moreover, between any vertex $v$ and himself, there cannot be an edge.

We say that two vertices are adjacent if $\{u, v\} \in E$, an edge $\{u, v\} \in E$ is always incident to its extremes. Any edge will be named, by convention, using the labels for its extremes e.g. $\{u, v\}$ is translated to $uv$.
\begin{definition}[Neighbour]
    Given $u, v \in V$ two adjacent vertices, $v$ is neighbour of $u$. The set of neighbours of $u$ is called the neighbourhood of $u$ and we denote it with $\ngbrs{u}$. The degree of $u$ is $\deg{u} = |\ngbrs{u}|$.
\end{definition}
\begin{definition}[Order]
    The order of $G$ is $n(G) = |V|$ and the size of the graph is $e(G) = |E|$.
\end{definition}
\begin{definition}[Subgraph]
    Given two graphs $G, H$: We say $H$ is a subgraph of $G$ if $V(H) \subseteq V(G)$ and $E(H) \subseteq E(G)$, we denote this by saying that $H \subseteq G$.
\end{definition}
\begin{definition}[Induced subgraph]
    A subgraph $H$ of $G$ is induced if, for any $u, v \in V(H), uv \in E(H) \iff u, v \in V(G)$
\end{definition}
For any $U \subseteq V(G)$ we write $G[U]$ for the subgraph $G$ induced by $U$, that is:

$V(G[U]) = U$

and

$E(G[U]) = \{vw \in E(G): v, w \in U\}$

We can also define the difference between two graphs as:
\begin{equation*}
    G - U = G[V(G) \setminus U]
\end{equation*}
Given $F \subseteq E(G)$, we write $G - F$ for the graph with vertex set $V(G)$ and edge set $E(G) \setminus F$.
\begin{definition}[Complement]
    The complement of $G$, denoted $\overline{G}$ is the graph on vertex set $V(G)$ and edge set $\binom{V}{2} \setminus E(G)$
\end{definition}
Usually in graph theory we derive three measures from the concept of vertex degree
\begin{definition}[Maximum degree]
    The maximum degree of a graph can be defined as follows:
    \begin{equation*}
        \maxd = \max_{v \in V(G)}\deg{v}
    \end{equation*}
\end{definition}
\begin{definition}[Average degree]
    The average degree of a graph can be defined as follows:
    \begin{equation*}
        \avgd = \frac{1}{|V(G)|}\sum_{v \in V(G)}\deg{v}
    \end{equation*}
\end{definition}
\begin{definition}[Minumum degree]
    The minimum degree of a graph can be defined as follows:
    \begin{equation*}
        \mind = \min_{v \in V(G)}\deg{v}
    \end{equation*}
\end{definition}
\begin{definition}[$r$-regular graph]
    A graph is $r$-regular if $\deg{v} = r \fa v \in V(G)$. $G$ is just regular if it is regular for some $r \in \N$
\end{definition}
Now we will prove a very important lemma
\begin{lemma}[Handshaking Lemma]
    Let $G$ be a graph, then $2e(G) = \sum_{v \in V(G)}\deg{v}$ and $G$ has an even number of odd degree vertices.
\end{lemma}
\begin{prf}
    Let $S$ be the set of pairs $(v, e)$ where $v \in V(G)$ and $e \in E(G)$, which is incident to $v$.

    For any $v \in V(G)$, the number of pairs of the form $(v, e) \in S$ is $\deg{v}$, therefore:
    \begin{equation*}
        |S| = \sum_{v \in V(G)}\deg{v}
    \end{equation*}
    For any edge $e \in E(G)$, the number of pairs of the form $(v, e) \in S$ is $2$, thus
    \begin{equation*}
        |S| = 2|E(G)|
    \end{equation*}
    The two expressions have to be the same, this implies that
    \begin{equation*}
        \sum_{v \in V(G)}\deg{v} = 2|E(G)|
    \end{equation*}
    Since $\sum_{v \in V(G)}\deg{v}$ is even, then the sum contains an even number of odd terms. \qed
\end{prf}
\begin{proposition}
    Any graph $G$ is the subgraph of a $\maxd$-regular graph.
\end{proposition}
\begin{prf}
    We reason by induction on $k = \maxd - \mind$

    \boldmath $k = 0$ \unboldmath

    $G$ is regular and $G \subseteq G$
    
    \boldmath $k > 0$ \unboldmath

    Let $G'$ be a disjoint copy of $G$, let $W$ and $W'$ be the sets of minimum degree vertices in $G$ and $G'$, respectively.

    For any vertex $w \in W$, add an edge from $w$ to the corresponding vertex $w' \in W'$.

    Iterate the process until all the corresponding vertices from $W$ and $W'$ have been linked by an edge, we can conclude the following:

    Let $H$ be the resulting graph:
    \begin{itemize}
        \item $G \subseteq H$
        \item For each $w \in W$ we have $\deg[H]{w} = \deg{w} + 1 = \mind + 1$
        \item For each $w' \in W'$ we have $\deg[H]{w'} = \deg[G']{w} + 1 = \mind[G'] + 1$
        \item For each $v \in V(H) \setminus (W \cup W')$ we have:
        \begin{equation*}
            \deg[H]{v} =
            \begin{cases}
                &\deg{v} \text{\hspace{5mm}if } v \in G\\
                &\deg[G']{v} \text{\hspace{5mm}if } v \in G'
            \end{cases}
        \end{equation*}
    \end{itemize}
    Thus $\maxd[H] = \maxd$ and $\mind[H] = \mind + 1$
    By the induction hypthesis there is a graph $H'$ such that $H'$ is $\maxd[H]$-regular and $H \subseteq H' \implies G \subseteq H \subseteq H'$ and $H'$ is $\maxd$-regular \qed
\end{prf}
\begin{definition}[Edge density]
    $\ed = \frac{e(g)}{n(G)} = \frac{\avgd}{2}$
\end{definition}
\begin{proposition}
    Any graph $G$, with at least one edge, has a subgraph $H$ with
    \begin{equation*}
        \mind > \ed[H] \geq \ed
    \end{equation*}
\end{proposition}
\begin{prf}
    We will do an induction proof on $n(G)$, the \textbf{base case} is the one in which we have a single edge
    \begin{figure}[h]
        \centering
        \begin{tikzpicture}
    \node[defv] (n1) at (0,0) {$v_1$};
    \node[defv] (n2) at (3,0) {$v_2$};

    \draw[defe] (n1) edge node[above] {} (n2);
\end{tikzpicture}
    \end{figure}
    The above is called $\clq{2}$ and it's the complete graph containing all of the edges connecting 2 vertices ($v_1$ and $v_2$ in our case).

    In this case we are having:
    \begin{align*}
        \ed[\clq{2}] &= \frac{1}{2} \\
        \mind[\clq{2}] &= 1
    \end{align*}
    Let $H = \clq{2}$, every graph is subgraph of itself.
    In the \textbf{inductive case}, $n(G) > 2$, we suppose that for any $G'$ with $n(G') < n(G)$ and $E(G') \geq 1, \ex H \subseteq G'$ such that $\mind[H] > \ed[H] \geq \ed[G']$.
    
    Removing vertices with low degree increases the edge density. Let $v \in V(G)$ with $\deg{v} \leq \ed = \frac{m}{n}$

    Then if we remove the vertex
    \begin{equation*}
        \ed[G - v] = \frac{e(G - v)}{n - 1} = \frac{m - \deg{v}}{n - 1} \geq \frac{m - \frac{m}{n}}{n - 1} = \frac{m}{n} \cdot \frac{n - 1}{n - 1} = \frac{m}{n} = \ed
    \end{equation*}
    By the induction hypothesis $\ex H \subseteq G - v$ such that
    \begin{equation*}
        \mind[H] > \ed[H] \geq \ed[G - v] \geq \ed
    \end{equation*}
    And that proves our proposition.\qed
\end{prf}
Let $G$ and $G'$ be two graphs, we say that $G$ and $G'$ are isomorphic ($G \cong G'$) if there is a bijection $\phi: V(G) \rightarrow V(G')$ where, for any $u, v \in V(G), uv \in E(G) \iff \phi(u)\phi(v) \in E(G')$. $\phi$ is called isomorphism, if domain and codomain are the same then the function is colled an automorphism.
\begin{cf}
    The \prblm{Graph isomorphism problem} consists in determining whether two graphs are isomorphic, and it is one of the problems suspected to be in between \P and \NP.
\end{cf}
\begin{definition}[Girth]
    The girth of a graph is the minimum length of a cycle in $G$ or $\infty$ if there are no cycles
\end{definition}
\begin{proposition}
    Let $G$ be a graph, then $G$ has a path of length $\mind$. Moreover $\mind \geq 2$, then $G$ has a cycle of length $\geq \mind + 1$
\end{proposition}
\begin{prf}
    Let $P = v_0\:v_1\:v_l$ be a path of maximum length in $G$, since $P$ is maximum, $\ngbrs{v_l} \subseteq V(P)$, otherwise we could extend $P$ (esamples of non maximal paths are shown in red in \ref{fig:proposition1.6}), so the length of $P$ is:
    \begin{figure}[h]
        \centering
        \begin{tikzpicture}
    \node[defv] (n1) at (0,0) {$v_1$};
    \node[defv] (n2) at (1.5,0) {$v_2$};
    \node[defv] (n3) at (3,0) {$v_3$};
    \node[defv] (n4) at (6,0) {$v_{l - 1}$};
    \node[defv] (n5) at (8,0) {$v_{l}$};

    \draw[defe] (n1) edge node[above] {} (n2);
    \draw[defe] (n2) edge node[above] {} (n3);
    \draw[defe] (n4) edge node[above] {} (n5);
    \draw[rede, out=135, in=30] (n5) edge node[above] {} (n3);
    \draw[rede, out=135, in=30] (n5) edge node[above] {} (n2);
    \draw[dotted] (n3) edge node[above] {} (n4);
\end{tikzpicture}
        \caption{Example of non maximal paths}
        \label{fig:proposition1.6}
    \end{figure}
    \begin{equation*}
        l \geq |\ngbrs{v_l}| \geq \mind
    \end{equation*}
    Now suppose $\mind \geq 2$.

    Let $i$ be the first index such that $v_iv_l \in E(G) \implies \ngbrs{v_l} \subseteq \{v_i, \dots, v_{l - 1}\}$, since $\mind \geq 2, i \neq l - 1$. So $v_i \dots v_lv_i$ is a cycle of length $\geq \mind + 1$ \qed
\end{prf}
\begin{definition}[Diameter]
    The diameter of a graph $G$ is the length of the greatest shortest path between any two vertices in $G$
\end{definition}
\begin{proposition}
    For any cyclic graph:
    \begin{equation*}
        g(G) \geq 2diam(G) + 1
    \end{equation*}
\end{proposition}
\begin{prf}
    Let $\cycl$ be a cycle of minimum length, suppose for a contradiction that it has length $\geq 2diam(G) + 2$.

    Thus there are two vertices $u, v \in V(\cycl)$ such that $dist_{\cycl}(u, v) \geq diam(G) + 1$. Therefore $G$ contains a $u - v$ path $P$ of length $\leq diam(G)$, $P$ is not entirely contained in the cycle, so there exist two vertices $x, y \in P \cap V(\cycl)$ such that $E(xPy) \cap E(\cycl) = \emptyset$.

    Note that $xPy$ has length $\leq diam(G) < \frac{e(\cycl)}{2}$ so that together with the shortest $x - y$ path on $\cycl$, $xPy$ gives a cycle of length $< e(\cycl)$ which is contradictory. \qed
\end{prf}