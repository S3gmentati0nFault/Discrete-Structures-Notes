\chapter{The basics}
\begin{definition}[Graph]
    A graph is a pair $(V, E)$ of finite disjoint sets where:
    \begin{itemize}
        \item $V$ is the set of vertices
        \item $E$ is a set of pairs, $E \subseteq \binom{V}{2}\footnote{The binomial coefficient has an interpretation in the field of combinatorics: it gives the number of ways, disregarding order, that $k$ objects can be chosen among $n$ objects. More formally: it defines the number of $k$-element subsets of an $n$-element set.} = \grf{(u, v): u \neq v, \text{with } u, v \in V}.$
    \end{itemize}
\end{definition}
As a default, for the rest of the course, we will be considering simple graphs, which means that between any two $u, v$ vertices, there cannot be more than one edge. Moreover, between any vertex $v$ and himself, there cannot be an edge.

We say that two vertices are adjacent if $\grf{u, v} \in E$, an edge $\grf{u, v} \in E$ is always incident to its extremes. Any edge will be named, by convention, using the labels for its extremes e.g. $\grf{u, v}$ is translated to $uv$.
\begin{definition}[Neighbour]
    Given $u, v \in V$ two adjacent vertices, $v$ is neighbour of $u$. The set of neighbours of $u$ is called the neighbourhood of $u$ and we denote it with $\ngbrs{u}$. The degree of $u$ is $\deg{u} = |\ngbrs{u}|$.
\end{definition}
\begin{definition}[Order]
    The order of $G$ is $\nv = |V|$ and the size of the graph is $\ne = |E|$.
\end{definition}
\begin{definition}[Subgraph]
    Given two graphs $G, H$: We say $H$ is a subgraph of $G$ if $\v[H] \subseteq \v$ and $\e[H] \subseteq \e$, we denote this by saying that $H \subseteq G$.
\end{definition}
\begin{definition}[Induced subgraph]
    A subgraph $H$ of $G$ is induced if, for any $u, v \in \v[H], uv \in \e[H] \iff u, v \in \v$
\end{definition}
\begin{figure}[h]
    \centering
    \begin{tikzpicture}
    \node[defv] (n1) at (0,0) {$v_1$};
    \node[defv] (n2) at (2,0) {$v_2$};
    \node[defv] (n3) at (2,2) {$v_3$};
    \node[defv] (n4) at (0,2) {$v_4$};
    \node[defv] (n5) at (4,2) {$v_5$};

    \draw[defe] (n1) edge node[above] {} (n2);
    \draw[defe] (n2) edge node[above] {} (n3);
    \draw[defe] (n4) edge node[above] {} (n1);
    \draw[defe] (n3) edge node[above] {} (n4);
    \draw[defe] (n3) edge node[above] {} (n5);

    \draw[ultra thick, rede, dashed] (n1) edge node[above] {} (n2);
    \draw[ultra thick, rede, dashed] (n2) edge node[above] {} (n3);
    \draw[ultra thick, rede, dashed] (n4) edge node[above] {} (n1);
    \draw[ultra thick, rede, dashed] (n3) edge node[above] {} (n4);
\end{tikzpicture}
    \label{fig:induced_subgraph}
    \caption{In red an example of induced subgraph}
\end{figure}
For any $U \subseteq \v$ we write $G[U]$ for the subgraph $G$ induced by $U$, that is:
\begin{align*}
    \v[{G[U]}] &= U \\
    \e[{G[U]}] &= \grf{vw \in \e: v, w \in U}
\end{align*}
We can also define the difference between two graphs as:
\begin{equation*}
    G - U = G[\v \setminus U]
\end{equation*}
Given $F \subseteq \e$, we write $G - F$ for the graph with vertex set $\v$ and edge set $\e \setminus F$.
\begin{definition}[Complement]
    The complement of $G$, denoted $\overline{G}$ is the graph on vertex set $\v$ and edge set $\binom{V}{2} \setminus \e$
\end{definition}
Usually in graph theory we derive three measures from the concept of vertex degree
\begin{definition}[Maximum degree]
    The maximum degree of a graph can be defined as follows:
    \begin{equation*}
        \maxd = \max_{v \in \v}\deg{v}
    \end{equation*}
\end{definition}
\begin{definition}[Average degree]
    The average degree of a graph can be defined as follows:
    \begin{equation*}
        \avgd = \frac{1}{|\v|}\sum_{v \in \v}\deg{v}
    \end{equation*}
\end{definition}
\begin{definition}[Minumum degree]
    The minimum degree of a graph can be defined as follows:
    \begin{equation*}
        \mind = \min_{v \in \v}\deg{v}
    \end{equation*}
\end{definition}
\begin{definition}[$r$-regular graph]
    A graph is $r$-regular if $\deg{v} = r \fa v \in \v$. $G$ is just regular if it is regular for some $r \in \N$
\end{definition}
Now we will prove a very important lemma
\begin{customlemma}{Handshaking}
    \label{lemma:handshaking}
    Let $G$ be a graph, then $2\ne = \sum_{v \in \v}\deg{v}$ and $G$ has an even number of odd degree vertices.
\end{customlemma}
\begin{prf}
    For this proof we will be applying the double counting technique. Let $S$ be the set of pairs $(v, e)$ where $v \in \v$ and $e \in \e$, which is incident to $v$.

    For any $v \in \v$, the number of pairs of the form $(v, e) \in S$ is $\deg{v}$, therefore:
    \begin{equation*}
        |S| = \sum_{v \in \v}\deg{v}
    \end{equation*}
    For any edge $e \in \e$, the number of pairs of the form $(v, e) \in S$ is $2$, thus
    \begin{equation*}
        |S| = 2|\e|
    \end{equation*}
    The two expressions have to be the same, this implies that
    \begin{equation*}
        \sum_{v \in \v}\deg{v} = 2|\e|
    \end{equation*}
    Since $\sum_{v \in \v}\deg{v}$ is even, then the sum contains an even number of odd terms. \qed
\end{prf}
\begin{customproposition}{1.4}
    \label{proposition:1.4}
    Any graph $G$ is the subgraph of a $\maxd$-regular graph.
\end{customproposition}
\begin{prf}
    We reason by induction on $k = \maxd - \mind$

    \boldmath $k = 0$ \unboldmath

    $G$ is regular and $G \subseteq G$
    
    \boldmath $k > 0$ \unboldmath

    Let $G'$ be a disjoint copy of $G$, let $W$ and $W'$ be the sets of minimum degree vertices in $G$ and $G'$, respectively.

    For any vertex $w \in W$, add an edge from $w$ to the corresponding vertex $w' \in W'$.

    Iterate the process until all the corresponding vertices from $W$ and $W'$ have been linked by an edge, we can conclude the following:

    Let $H$ be the resulting graph:
    \begin{itemize}
        \item $G \subseteq H$
        \item For each $w \in W$ we have $\deg[H]{w} = \deg{w} + 1 = \mind + 1$
        \item For each $w' \in W'$ we have $\deg[H]{w'} = \deg[G']{w} + 1 = \mind[G'] + 1$
        \item For each $v \in \v[H] \setminus (W \cup W')$ we have:
        \begin{equation*}
            \deg[H]{v} =
            \begin{cases}
                &\deg{v} \text{\hspace{5mm}if } v \in G\\
                &\deg[G']{v} \text{\hspace{5mm}if } v \in G'
            \end{cases}
        \end{equation*}
    \end{itemize}
    Thus $\maxd[H] = \maxd$ and $\mind[H] = \mind + 1$
    By the induction hypthesis there is a graph $H'$ such that $H'$ is $\maxd[H]$-regular and $H \subseteq H' \implies G \subseteq H \subseteq H'$ and $H'$ is $\maxd$-regular \qed
\end{prf}
\begin{definition}[Edge density]
    $\ed = \frac{\ne}{\nv} = \frac{\avgd}{2}$
\end{definition}
\begin{customproposition}{1.5}
    \label{proposition:1.5}
    Any graph $G$, with at least one edge, has a subgraph $H$ with
    \begin{equation*}
        \mind > \ed[H] \geq \ed
    \end{equation*}
\end{customproposition}
\begin{prf}
    We will do an induction proof on $\nv$, the \textbf{base case} is the one in which we have a single edge as shown in \ref*{fig:k_2}. $K_2$ is the complete graph containing all of the edges connecting 2 vertices.
    \begin{figure}[h]
        \centering
        \begin{tikzpicture}
    \node[defv] (n1) at (0,0) {$v_1$};
    \node[defv] (n2) at (3,0) {$v_2$};

    \draw[defe] (n1) edge node[above] {} (n2);
\end{tikzpicture}
        \caption{A clique composed of just two vertices}
        \label{fig:k_2}
    \end{figure}

    In this case we are having:
    \begin{align*}
        \ed[\clq{2}] &= \frac{1}{2} \\
        \mind[\clq{2}] &= 1
    \end{align*}
    Let $H = \clq{2}$, every graph is subgraph of itself.
    In the \textbf{inductive case}, $\nv > 2$, we suppose that for any $G'$ with $n(G') < \nv$ and $\e[G'] \geq 1, \ex H \subseteq G'$ such that $\mind[H] > \ed[H] \geq \ed[G']$.
    
    Removing vertices with low degree increases the edge density. Given a graph $G$, let $v \in \v$ with $\deg{v} \leq \ed = \frac{m}{n}$

    Then we call $G' = G - v$, therefore
    \begin{equation*}
        \ed[G'] = \frac{e(G')}{n - 1} = \frac{m - \deg{v}}{n - 1} \geq \frac{m - \frac{m}{n}}{n - 1} = \frac{m}{n} \cdot \frac{n - 1}{n - 1} = \frac{m}{n} = \ed
    \end{equation*}
    By the induction hypothesis $\ex H \subseteq G'$ such that
    \begin{equation*}
        \mind[H] > \ed[H] \geq \ed[G'] \geq \ed
    \end{equation*}
    And that proves our proposition.\qed
\end{prf}
Let $G$ and $G'$ be two graphs, we say that $G$ and $G'$ are isomorphic ($G \cong G'$) if there is a bijection $\phi: \v \rightarrow V(G')$ where, for any $u, v \in \v, uv \in \e \iff \phi(u)\phi(v) \in \e[G']$. $\phi$ is called isomorphism, if domain and codomain are the same then the function is colled an automorphism.
\begin{coolfact}
    The \prblm{Graph isomorphism problem} consists in determining whether two graphs are isomorphic, and it is one of the problems suspected to be in between \P and \NP.
\end{coolfact}
\begin{definition}[Girth]
    The girth $g(\smth)$ of a graph is the minimum length of a cycle in $G$ or $\infty$ if there are no cycles
\end{definition}
\begin{customproposition}{1.6}
    \label{proposition:1.6}
    Let $G$ be a graph, then $G$ has a path of length $\geq \mind$. Moreover if $\mind \geq 2$, then $G$ has a cycle of length $\geq \mind + 1$
\end{customproposition}
\begin{prf}
    \begin{figure}[h]
        \centering
        \begin{tikzpicture}
    \node[defv] (n1) at (0,0) {$v_1$};
    \node[defv] (n2) at (1.5,0) {$v_2$};
    \node[defv] (n3) at (3,0) {$v_3$};
    \node[defv] (n4) at (6,0) {$v_{l - 1}$};
    \node[defv] (n5) at (8,0) {$v_{l}$};

    \draw[defe] (n1) edge node[above] {} (n2);
    \draw[defe] (n2) edge node[above] {} (n3);
    \draw[defe] (n4) edge node[above] {} (n5);
    \draw[rede, out=135, in=30] (n5) edge node[above] {} (n3);
    \draw[rede, out=135, in=30] (n5) edge node[above] {} (n2);
    \draw[dotted] (n3) edge node[above] {} (n4);
\end{tikzpicture}
        \caption{Example of non maximal paths}
        \label{fig:proposition1.6}
    \end{figure}
    Let $P = v_0\:v_1\dots\:v_l$ be a path of maximum length in $G$, since $P$ is maximum, $\ngbrs{v_l} \subseteq V(P)$, otherwise we could extend $P$ (esamples of non maximal paths are shown in red in \ref{fig:proposition1.6}), for the same reason the length of the path is:
    \begin{equation*}
        l \geq |\ngbrs{v_l}| \geq \mind
    \end{equation*}
    Now suppose $\mind \geq 2$.

    Let $i$ be the first index such that $v_iv_l \in \e \implies \ngbrs{v_l} \subseteq \grf{v_i, \dots, v_{l - 1}}$, since $\mind \geq 2, i \neq l - 1$. So $v_i \dots v_lv_i$ is a cycle of length $\geq \mind + 1$ \qed
\end{prf}
\begin{definition}[Diameter]
    The diameter of a graph $diam(\smth)$ is the greatest shortest path in the graph.
\end{definition}
\begin{customproposition}{1.7}
    \label{proposition:1.7}
    For any cyclic graph:
    \begin{equation*}
        g(G) \geq 2diam(G) + 1
    \end{equation*}
\end{customproposition}
\begin{prf}
    Let $\cycl$ be a cycle of minimum length, suppose for a contradiction that it has length $\geq 2diam(G) + 2$.

    Thus there are two vertices $u, v \in V(\cycl)$ such that $dist_{\cycl}(u, v) \geq diam(G) + 1$. Therefore $G$ contains a $u - v$ path $P$ of length $\leq diam(G)$, $P$ is not entirely contained in the cycle, so there exist two vertices $x, y \in P \cap V(\cycl)$ such that $\e[xPy] \cap \e[\cycl] = \emptyset$.

    Note that $xPy$ has length $\leq diam(G) < \frac{e(\cycl)}{2}$ so that together with the shortest $x - y$ path on $\cycl$, $xPy$ gives a cycle of length $< e(\cycl)$ which is contradictory. \qed
\end{prf}
\begin{customproposition}{1.8}
    \label{proposition:1.8}
    For $d \in \R$ and $g \in \N$, let
    \begin{equation*}
        n_0 =
        \begin{cases}
            &1 + d \sum_{i = 0}^{r - 1}(d - 1)^i \hspace{5mm} \text{if $g =: 2r + 1$ is odd}\\
            &2\sum_{i = 0}^{r - 1}(d - 1)^i \hspace{5mm} \text{if $g =: 2r$ is even}
        \end{cases}
    \end{equation*}
\end{customproposition}
Let $\delta, g \geq 3$ and $G$ a graph with $\mind \geq \delta$ and $g(G) \geq g$ then $\nv \geq n_0(\delta, g)$.\\
\begin{coolfact}
    There was a result by Alan, Hooray and Linial which replaced $\mind$ with $\avgd$
\end{coolfact}
\begin{customcorollary}{1.9}
    \label{corollary:1.9}
    If $\mind \geq 3$ then $g(G) < 2\log{n}$
\end{customcorollary}
\begin{customlemma}{1.11}
    \label{lemma:1.11}
    Let $G$ be a graph, the relation $\conn$ defined by:
    \begin{equation*}
        x \conn y \iff \exists x-y \text{path in } G
    \end{equation*}
    For $x, y \in \v$ is an equivalence relation.
\end{customlemma}
Since the proof is trivial we will not be showing it. Having defined an equivalence relation we can see different connected components as equivalence classes, more formally:
\begin{definition}[Component]
    The components of a graph $G$ are its maximal connected subgraphs, i.e. the subgraphs induced by the equivalence classes.
\end{definition}
\begin{definition}[$k$-connectivity]
    Let $k \in \N$, then $G$ is $k$-connected if $\nv \geq k + 1$ and for any $X \subseteq \v$ with $|X| < k$, the graph $G - X$ is connected.
\end{definition}
Let's think about a couple of examples:
\begin{itemize}
    \item A $1$-connected graph is a simply connected graph. Removing one node will just break connectivity.
    \item Any cycle is at most $2$-connected.
\end{itemize}
The largest $k$ such that $G$ is $k$-connected is called connectivity and we denote it with $k(G)$.
\begin{definition}[Separator]
    A set $X \subseteq \v$ such that $G - X$ is disconnected is called a separator.
\end{definition}
\begin{customtheorem}{(Mader)}
    \label{theorem:mader}
    Any graph $G$ with average degree $\geq 4k, k \in \N$, has a subgraph $H$ which is $(k + 1)$-connected and satisfies $\ed[H] > \ed - k$
\end{customtheorem}
\begin{prf}
    Let $\varepsilon \das \ed$, now choose $H$ to be a subgraph of $G$ satisfying the following
    \begin{itemize}
        \item $n(H) \geq 2k$
        \item $e(H) > \varepsilon(n(H) - k)$
    \end{itemize}
    with $n(H)$ minimum. $H$ exists since $G$ itself satisfies the above, simply because
    \begin{itemize}
        \item $\nv \geq \avgd \geq 2k$
        \item $e(G) = \varepsilon \cdot \nv > \varepsilon (\nv - h)$
    \end{itemize}
    Before proceding we want to prove a couple of claims.
    \begin{claim}
        \label{claim:1_Mader}
        \begin{equation*}
            n(H) > 2k
        \end{equation*}
    \end{claim}
    \begin{prf}
        Suppose not, that means $n(H) = 2k$, thus $e(H) > \varepsilon(2k - k) \geq 2k^2$

        But $\ne[H] \leq \ne[K_{2k}] = \binom{2k}{2} < 2k^2$

        Which is absurd, no graph can have more edges than a complete graph \qed
    \end{prf}
    \begin{claim}
        \label{claim:2_Mader}
        \begin{equation*}
            \mind[H] \geq \varepsilon
        \end{equation*}
    \end{claim}
    \begin{prf}
        Suppose not, let $v \in \v[H]$ with $\deg[H]{v} \leq \varepsilon$. Then 
        \begin{equation*}
            \ne[H - v] \geq \ne[H] - \varepsilon > \varepsilon(\nv[H] - 1 - k) = \varepsilon(\nv[H - v] - k)
        \end{equation*}
        Thus the property still holds, which means that contradicts the minimality of $H$ and that is absurd. \qed
    \end{prf}
    Thus:
    \begin{equation*}
        \ed[H] = \frac{\ne[H]}{\nv[H]} > \frac{\ed[\nv[H] - k]}{\nv[H]} = \varepsilon - \frac{\varepsilon_k}{\nv[H]} > \varepsilon - \frac{\varepsilon_k}{\mind[H]} > \varepsilon - \frac{\varepsilon_k}{\varepsilon} = \varepsilon - k
    \end{equation*}
    It remains to show that $H$ is $k + 1$ conneted, by \textbf{Claim 1} \ref{claim:1_Mader} we have
    \begin{equation*}
        \nv > 2k \geq k + 2
    \end{equation*}
    The last thing that we have to prove is that there is no separator of size $\leq k$.

    Let's suppose for a contradiction that there is a separator of size $\leq k$ for $H$, we will call it $X$, $H_1$ and $H_2$ must be non-empty, thus they must be proper subgraphs. Let $k$ be a component of $H - X$ and define
    \begin{align*}
        H_1 &= H[\v[K] \cup X]\\
        H_2 &= H - \v[K]
    \end{align*}
    By minimality of $H, H_1, H_2$ violate \ref{}

    Vertices in $H$ have the same degree as in $H_1$ thus we want to show that both
    \begin{equation*}
        \nv[H_1], \nv[H_2] \geq 2k
    \end{equation*}
    \begin{equation*}
        \varepsilon = \frac{\ne[H]}{\nv[H]}
    \end{equation*}
    We recall
    \begin{equation*}
        \avgd[H] = \frac{\sum\avgd[V]}{\nv[H]} = \frac{2\ne[H]}{\nv[H]}
    \end{equation*}
    Note that all vertices in $K$ have their neighbours only in $K \cup X$ (otherwise $K$ would not be a component of $H - X$). $\fa v \in \v[K]$ we have $\deg[H_1]{v} = \deg[H]{v} > \varepsilon$, that is true because of \textbf{Claim 2} \ref{claim:2_Mader}, moreover we can say that since $\varepsilon \geq 2k \implies \deg[H_1]{v} \geq 2k$.
    
    Similarly, since $H_2 - X$ is non-empty (otherwise $H - X$ would only consist of $K$ and that would make it connected), there exists $x \in \v[H_2] \setminus X$ such that $\deg[H_2]{x} = \deg[H]{x} \geq 2k \implies \nv[H] \geq 2k$

    Since $H_1$ and $H_2$ violate \ref{}, we have $\ne[H_i] \leq \ed[\nv[H_i] - K] \fa i \in [Z]$, since every edge is connected at least once
    \begin{align*}
        \ne[H] &\leq \ne[H_1] + \ne[H_2]\\
        \ne[H] &\leq \varepsilon(\nv[H_1] - k + \nv[H_2] - k)\\
        &= \varepsilon(\nv[H] + |X| - 2k) \leq \varepsilon(\nv[H] - k)
    \end{align*}
    This violates \ref{}, so there is no such $X$, thus no separator of size $\leq k$, which means that $H$ is $(k + 1)$ connected. \qed
\end{prf}
\begin{customtheorem}{(Carmensin)}
    \label{theorem:carmesin}
    Evey graph $G$ with $\avgd \geq (3 + \frac{1}{3})k$ has a $k + 1$-connected subgraph with more than $2k$ vertices.
\end{customtheorem}
We can extend the concept of connetivity to edges, we say that a graph with at least $2$ vertices is $k$-edge-connected if removing less than $k$ edges does not disconnect the graph. Any cycle, for example, other than being $2$-connected, is also $2$-edge connected.
\begin{definition}[Edge-connectivity]
    The edge-connectivity of a graph $G$ is the maximum $k$ such that the graph is $k$-edge-connected.    
\end{definition}