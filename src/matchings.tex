\chapter{Matchings}
\begin{definition}[Matching]
    A matching is a set of vertex-disjoint edges.    
\end{definition}
A matching in $G$ is maximum if no other matching in $G$ has more edges. A very important detail needs to be clarified at this point: \textbf{maximum} $\neq$ \textbf{maximal}, explained easily:
\begin{itemize}
    \item [\textbf{Maximum:}] The greatest in size
    \item [\textbf{Maximal:}] It cannot be extended
\end{itemize}
This means that:
\begin{equation}
    \text{\textbf{Maximum}} \implies \text{\textbf{Maximal}} \:\:\text{but}\:\: \text{\textbf{Maximal}} \nimplies \text{\textbf{Maximum}}
\end{equation}
Let $G$ be a graph and $M$ a matching in $G$, an augmenting path in $G$ with respect to $M$ is a path in $G$ which starts at an unmatched vertex and alternates between edges of $\e \setminus M$ and $M$, and ends at an unmatched vertex. Note that $\e[P] \setminus M$ is a matching containing one edge more than $M$.
\begin{customlemma}{2.1}
\label{lemma:2.1}
    A matching is maximum $\iff$ it has no augmenting path.
\end{customlemma}
An alternating path is a path starting at an unmatched vertex and alternates between edges of $\e \setminus M$ and $M$.
\section{Bipartite Matchings}
\begin{definition}[Vertex cover]
    A vertex cover is a set $U$ of vertices such that each edge of $G$ has an endpoint in $U$.
\end{definition}
\begin{customtheorem}{(König)}
\label{theorem:konig}
    Let $G$ be a bipartite graph: the maximum size of a matching in $G$ is the maximum size of a vertex cover of $G$.
\end{customtheorem}
\begin{prf}
    Let $M$ be a maximum matching and $U$ be a minimum vertex cover. If $|M| \leq |U|:\:U$ is a vertex cover, which means that for each $e \in M$ one of the endpoints of $e$ is in $U$. Since the edges in $M$ are disjoint, we need $|M|$ vertices in $V$ to cover $M$.

    Denote by $(A, B)$ the bipartition of $G$, construct a set $U \subseteq \v$ as follows: For each $a, b \in M$, where $a \in A$ and $b \in B$ if $G$ has an alternating path starting in $A$ and ending in $b$ then add $b$ to $U$, otherwise we add $a$ to $U$. By construction $|M| = |U|$. It remains to show that $U$ is a vertex cover.

    By definition of $U$, each $e \in M$ has an endpoint in $U$, so let $ab \in \e \setminus M, a \in A, b \in B^*$. For a contradiction, suppose both $a, b \notin U$, since $M$ is maximum, at least one between $a$ and $b$ is incident to an edge of $M$. If there is an alternating path $P$ starting in $A$ and ending at $b$, then, by definition of $U$, $b$ cannot be incident to an edge of $M$, so $P$ is an augmenting path. Since $P$ is an augmenting path then $M$ is not maximum, but that is a contradiction of the fact that $M$ is maximum (by Lemma \ref{lemma:2.1}).

    If $a$ is unmatched, then $ab$ would be such an alternating path. So $\ex ab' \in M$ since $a \notin U$ by assumption, there exists an alternating path $Q$ which starts in $A$ and ends at $b'$. If $a, b \notin \v[Q]$ then $Qb'ab$ would be an alternating path starting in $A$ and ending at $b$, which is absurd. If $b \in \v[Q]$ then $a'Qb$ is an alternating path, which is absurd.

    If $a \in \v[Q]$, then $a$ is matched, it must be an internal vertex of $Q$ and so since $Q$ is alternating $ab'$ must be in $Q$. Since $G$ is bipartite and the first vertex of $Q$ is unmatched, we must have that $b'$ is also an internal vertex. Which is absurd.\qed
\end{prf}
Given a graph $G$ and $A \subseteq \v$, a matching of $A$ in $G$ is a matching $M$ such that every $a \in A$ is incident to an edge of $M$. For $X \subseteq \v, \ngbrs{X} = \bigcup_{x \in X} \ngbrs{x} \setminus X$.
\begin{customtheorem}{(Hall)}
\label{theorem:hall}
    Let $G = (A \cup B, E)$ be a bipartite graph. Then $G$ has a matching of $A$ if an only if
    \begin{equation*}
        |\ngbrs{X}| \geq |X| \text{for all $X \subseteq A$}
    \end{equation*}
\end{customtheorem}
\begin{prf}
    To prove the theorem we will prove the two directions separately
    \begin{itemize}
        \item [($\implies$)] Let $M$ be a matching of $A$ note that for each $X \subseteq A$, each $x \in X$ is matched to a different $b_x \in B$.
        \item [($\impliedby$)] $|\ngbrs{X}| \geq |\{b_x \st x \in X\} = |X|$
        
        By Theorem \ref{theorem:konig}, $G$ has a matching $M$ and a vertex cover $U$ of the same cardinality. We want to show that $M$ is a matching of $A$.

        Since $|M| = |U|, U$ consists of precisely one endpoint of each edge of $M$. Let $A_U = A \cap U$ and $B_U = B \cap U$ and $A' = \{a \in A \st \ex b \in B_U \st ab \in M\}$. By the observation above, $U = A_U \cap B_U$ and $A_U \cap A' \neq \emptyset$, moreover each vertex is in $A_U \cup A'$ is incident to an edge of $M$. And $|A'| = |B_U|$ thus
        \begin{equation*}
            |A \setminus A'| \leq |\ngbrs{A \setminus A_U}| \leq |B_U|
        \end{equation*}
        but $|B_U| = |A'| \leq |A \setminus A_U|$, thus the equality holds throughout this string of inequalities. In particular, $|A'| = |A \setminus A_U|$. So $A = A_U \cup A'$ and, as already observed, each vertex of $A$ is matched.\qed
    \end{itemize}
\end{prf}
Let $G$ be a graph. A matching $M$ in $G$ is perfect if $\v[M] = \v$. A perfect matching is also called a $1$-factor. More generally, a $k$-factor in $G$ is a spanning $k$-regular subgraph in $G$.
\begin{customcorollary}{2.4}
\label{corollary:2.4}
    Let $k \geq 1$ and $k$-regular bipartite graph has a perfect matching.
\end{customcorollary}
\begin{prf}
    Let $(A, B)$ be the bipartition of $G$. Since $G$ is $k$-regular then $k \cdot |A| = \ne = k \cdot |B|$.
    \begin{equation*}
        k \cdot |A| = \sum_{v \in A}\deg{v} = \ne = \sum_{v \in B}\deg{v} = k \cdot |B|
    \end{equation*}
    So a matching of $A$ is a perfect matching, thus by Theorem \ref{theorem:hall} it is enough to check Hall's condition. Let $X \subseteq A$. If $|X| \leq k$, then the degree condition implies $|\ngbrs{x}| \geq k \geq |X|$. Suppose $|X| > k$
    \begin{equation*}
        |X| \cdot k = \ne[G[X \cup \ngbrs{X}]] \leq k \cdot |\ngbrs{x}|\\
        |\ngbrs{X}| \geq |X|
    \end{equation*}
    And that proves the theorem.\qed
\end{prf}