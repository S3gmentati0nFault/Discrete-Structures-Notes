\section{Bipartite graphs}
\begin{definition}[Bipartite Graph]
    A graph $G$ is bipartite if there is a partition $A \cup B = \v$ such that all edges have one extreme in $A$ and the other in $B$.
\end{definition}
We say $(A, B)$ is a bipartition of $G$, we denote by $\clq{m, n}$ the complete bipartite graph with bipartition $(A, B)$ such that $|A| = m$ and $|B| = n$.
\begin{customproposition}{1.24}
    \label{proposition:1.24}
    A graph is bipartite $\iff$ it has no odd cycle
\end{customproposition}
\begin{prf}
    To prove this theorem we will have to prove both directions.
    \begin{itemize}
        \item [($\implies$)] Suppose $G$ is bipartite, odd cycles are not bipartite (we call $(A, B)$ the bipartition). Note that $(A, B)$ induces a bipartition of any $H \subseteq G$ thus no odd cycle can be a subgraph of $G$.
        \item [($\impliedby$)] We can assume without loss of generality\footnote{What follows holds because we can look at every component separately and consider each one of them a connected component on its own, since the connected components of the graph form a partition of the graph itself and we can define a bipartition on each one of them, putting the bipartitions together we can actually compute the bipartition of the graph itself $G$. Thus it being connected or not doesn't make a difference.} that $G$ is connected, so $G$ has a spanning tree $T$. Let $(A, B)$ be a bipartition for $T$.
        
        We claim that $(A, B)$ is also a bipartition for $G$. Suppose not and say, without loss of generality, that $\ex aa' \in \e$ with $a, a' \in A$. Let $P$ be a $a - a'$ path in $T$, then $P$ alternates between $A$ and $B$, thus having even length. This means that $P + aa'$ is actually an odd cycle. But that contradicts our hypothesis, thus $G$ must have bipartition $(A, B)$\qed
    \end{itemize}
\end{prf}
\subsection{Minors}
A graph $H$ is a minor of $G$, denoted $H \minr G$ if $H$ can be obtained from $G$ via a series of vertex and edge deletions and edge contractions. Formally, contracting $xy \in \e$ means that we obtain a graph $G'$ with vertex set:
\begin{equation*}
    \v[G'] = \v \setminus \grf{x, y} \cup \grf{V_{xy}} \text{ where } V_{xy} \notin \v
\end{equation*}
and edge set:
\begin{equation*}
    \e[G'] = \grf{e \in \e: |xy \cap e| < 2}
\end{equation*}
\begin{customproposition}{1.25}
    \label{proposition:1.25}
    $X \minr G \iff$ there exist disjoint sets $V_x \subseteq \v, x \in \v[X]$ such that:
    \begin{itemize}
        \item $\fa x \in V_x, G[V_x]$ is connected
        \item For each $xy \in \e[X]$, there exists an edge in $G$ between $V_x$ and $V_y$
    \end{itemize}
\end{customproposition}
\begin{prf}
    To prove this theorem we will have to prove both directions.
    \begin{itemize}
        \item [($\impliedby$)] Suppose $\ex \grf{V_x \subseteq \v \st x \in \v[X]}$ as in the statement.
        
        Firts, delete all vertices outside of $V_x$, then contract each $V_x$ to a single vertex $v_x$. Whenever $xy \in \e[X]$, we had a $V_x - V_y$ edge, so $v_xv_y$ is now an edge. So we obtain $X$ by deleting any extra edges $V_xV_z$ for $xz \notin \e[X]$
        \item [($\implies$)] Suppose $X \minr G$ and fix a set of edge deletions, vertex deletions and edge contractions to obtain $X$ from $G$. For each $x \in \v[X]$ set $\v[X]$ to be the set of vertices which are contracted into $x$ to obtain $X$. Note that these sets are disjoint and induce connected graphs.
        
        If $xy \in \e[X]$ at the end of the series of deletions and contractions then $xy$ comes from an edge between a vertex that was contracted into $x$ and a vertex that was contracted into $y$, therefore $U_x \in V_x$ and $U_y \in V_y$, as desired.\qed
    \end{itemize}
\end{prf}
The sets $V_x$ are called branch sets.
\begin{customproposition}{1.26}
    \label{proposition:1.26}
    $\minr$ defines a partial order on the set of all finite graphs.
\end{customproposition}
Let $G$ be a graph. A subdivision of $G$ is a graph obtained by replacing some edges by internally disjoint paths. The vertices of $G$ are called the branch vertices of the subdivision and the new vertices are called subdivding vertices.

If G contains a subdivision of $H$, then $H$ is a topological minor of $G$. If $H$ is a topological minor of $G$ then $H$ is a minor of $G$, but if $H$ is a minor of $G$ that doesn't necessarily imply that $H$ is a topological minor of $G$. When we have a subdivision the degree of the branch vertices in maintained.
\begin{customproposition}{1.27}
    \label{proposition:1.27}
    Suppose $X \minr G$ with $\maxd[X] \leq 3$, then $X$ is a topological minor of $G$.
\end{customproposition}
\begin{prf}
    Let $H$ be a minimal subgraph of $G$ which still contains $X$ as a minor. Let $\grf{V_x \subseteq \v[H] \st x \in \v[X]}$ be the branch sets guaranteed by Proposition \ref{proposition:1.25} then:
    \begin{itemize}
        \item $H[V_x]$ is minimally connected (by minimality of $H$) thus the induced subgraphs are trees by Theorem \ref{theorem:1.15}.
        \item Each leaf of $H[V_x]$ has a neighbour outside of $V_x$ (otherwise we could delete it and preserve $x$ as a minor).
        \item There is $\leq 1$ edge between any two $V_x$ and $V_y$.
        \item If $xy \notin \e[X]$ then $H$ ahs no $V_x - V_y$ path.
    \end{itemize}
    We claim that $H$ is a subdivision of $X$. For each $x \in \v[X]$, we choose the corresponding branch vertex in $V_x$ as follows:
    \begin{itemize}
        \item If $V_x$ sends $\leq 2$ edges outside, then $H[V_x]$ is a path $P$ and the edges going outside are incident to the endpoints of $P$ so take any $b_x \in V_x$.
        \item If $H[V_x]$ is a path $P$ but sends $3$ edges outside. Then let $b_x$ be the unique internal vertex of $P$ which sends an edge outside of $V_x$.
        \item If $H[V_x]$ is not a path. Note $H[V_x]$ has $3$ leaves sending an edge outside $V_x.$
    \end{itemize}
    Then take $b_x$ to be a vertex of $H[V_x]$ such that there exist internally disjoint paths from $b_x$ to each of the leaves. $b_x$ exists, pick a path $P$ between $2$ of the leaves. Let $b_x$ be the vertex of $P$ which is at the smallest distance to the other leaf. This ensures that the path between $b_x$ and this last leaf does not use any edge of $P$. \qed
\end{prf}