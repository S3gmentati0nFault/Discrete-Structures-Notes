\section{Trees and forests}
A forest is a graph without cycles and a tree is a connected graph without cycles.
\begin{lemma}
    Any tree with at least $2$ vertices must have at least two leaves.
\end{lemma}
\begin{prf}
    Let $P$ be a largest path, by maximality of $P$ the extremities do not have neighbours outside the path $P$ and have just one neighbour inside the path, since there are no cycles we have $N(v_1) = \{v_1\}$ and $N(v_l) = \{v_{l - 1}\}$
\end{prf}
\begin{theorem}
    Let $T$ be a graph, the following statements are equivalent:
    \begin{enumerate}
        \item $T$ is a tree
        \item $T$ is minimally connected, i.e. $T$ is connected but $T - e$ is disconnected $\fa e \in \e[T]$
        \item $T$ is maximally acyclic, i.e. $T$ has no cycle but for any non adjacent $x \neq y \in \v[T], T + xy$ has a cycle
        \item For any $u, v \in \v[T]$ there is a unique $u - v$ path in $T$
    \end{enumerate}
\end{theorem}
\begin{prf}
    \boldmath$1 \implies 2$\unboldmath

    Suppose $T$ is a tree, by definition $T$ is connected. Let $uv \in \e[T]$, if $T - uv$ is connected, then $T$ has a $u - v$ path not using $uv$. Now $P + uv$ would be a cycle, which is absurd because we have a tree.

    \boldmath$2 \implies 3$\unboldmath

    Let's suppose by contradiction $T$ is minimally connected but contains a cycle, if we take $u, v \in \v[\cycl]$ and $u \in N_{\cycl}(v)$ then we know $uv \in \e[\cycl]$ thus removing the edge from the set $E(\cycl)$ should disconnect the tree because it's minimally connected, but there is at least one pair of vertices $x, y \in \v[\cycl]$ (possibly $\{x, y\} = \{u, v\}$) such that:
    \begin{itemize}
        \item There is a $u - x$ path in $\cycl$ that doesn't contain $uv$
        \item There is a $x - y$ path in $\cycl$ that doesn't contain $uv$
        \item There is a $y - v$ path in $\cycl$ that doesn't contain $uv$
    \end{itemize}
    But since the connection relationship is transitive (by Lemma \ref{lemma:1.11}) then there is stilla $u - v$ path even after removing $uv$, that means that the graph was not actually disconnected by the removal of $uv$ therefore we conclude that a tree has to be maximally acyclic since it's also minimally connected

    \boldmath$3 \implies 4$\unboldmath

    Suppose $T$ is maximally acyclic, let $u, v \in \v[T]$. If $uv \in \e[T]$, then $uv$ is a $u - v$ path. Otherwise, by assumption, $T + uv$ has a cycle $\cycl$ using $uv$ (if I didn't use $uv$ there would already be another cycle). So $\cycl - uv$ is a $u - v$ path in $T$.

    Suppose $P$ and $Q$ are two distinct $u - v$ paths in $T$, let $x$ be the first vertex on $P$ which is incident in $P$ to an edge not in $Q$. Let $y$ be the first vertex on $P$ after $x$ which is also in $Q$. Then $xPy$ and $yQx$ do not share any edge, so $xPyQx$ is a cycle in $T$.

    \boldmath$4 \implies 1$\unboldmath

    Suppose there exists a unique $u - v$ path between any two $u, v \in \v[T]$. Then, in particular, $T$ is connected. Moreover, if $T$ contained a cycle $\cycl$, any two vertices $u, v \in \v[\cycl]$ would be joined by two different paths.\qed
\end{prf}
