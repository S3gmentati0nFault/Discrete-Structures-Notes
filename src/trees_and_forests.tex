\section{Trees and forests}
A forest is a graph without cycles and a tree is a connected graph without cycles.
\begin{customlemma}{1.14}
    \label{lemma:1.14}
    Any tree with at least $2$ vertices must have at least two leaves.
\end{customlemma}
\begin{prf}
    Let's consider a maximum path $P$ in $T$.

    $P$ must have two extremities that are not linked to anything, the path does not contain cycles because if it did it would contradict the hypothesis of us having a tree, furthermore the extremes of the path must have degree $1$ (just one edge coming in) because if they didn't it would mean that there is an edge going out, and therefore we would be contradicting the maximum path hypothesis, since we would be able to extend it.

    Since we said that the two extremities must have degree $1$ they must be leaves, and this proves the theorem. \qed
\end{prf}
\begin{customtheorem}{1.15}
    \label{theorem:1.15}
    Let $T$ be a graph, the following statements are equivalent:
    \begin{enumerate}
        \item $T$ is a tree
        \item $T$ is minimally connected, i.e. $T$ is connected but $T - e$ is disconnected $\fa e \in \e[T]$
        \item $T$ is maximally acyclic, i.e. $T$ has no cycle but for any non adjacent $x \neq y \in \v[T], T + xy$ has a cycle
        \item For any $u, v \in \v[T]$ there is a unique $u - v$ path in $T$
    \end{enumerate}
\end{customtheorem}
\begin{prf}
    \boldmath$1 \implies 2$\unboldmath

    Suppose $T$ is a tree, by definition $T$ is connected. Let $uv \in \e[T]$, if $T - uv$ is connected, then $T$ has a $u - v$ path not using $uv$. Now $P + uv$ would be a cycle, which is absurd because we have a tree.

    \boldmath$2 \implies 3$\unboldmath

    Let's suppose by contradiction $T$ is minimally connected but contains a cycle, if we take $u, v \in \v[\cycl]$ and $u \in N_{\cycl}(v)$ then we know $uv \in \e[\cycl]$ thus removing the edge from the set $E(\cycl)$ should disconnect the tree because it's minimally connected, but there is at least one pair of vertices $x, y \in \v[\cycl]$ (possibly $\grf{x, y} = \grf{u, v}$) such that:
    \begin{itemize}
        \item There is a $u - x$ path in $\cycl$ that doesn't contain $uv$
        \item There is a $x - y$ path in $\cycl$ that doesn't contain $uv$
        \item There is a $y - v$ path in $\cycl$ that doesn't contain $uv$
    \end{itemize}
    But since the connection relationship is transitive (by Lemma \ref{lemma:1.11}) then there is stilla $u - v$ path even after removing $uv$, that means that the graph was not actually disconnected by the removal of $uv$ therefore we conclude that a tree has to be maximally acyclic since it's also minimally connected

    \boldmath$3 \implies 4$\unboldmath

    Suppose $T$ is maximally acyclic, let $u, v \in \v[T]$. If $uv \in \e[T]$, then $uv$ is a $u - v$ path. Otherwise, by assumption, $T + uv$ has a cycle $\cycl$ using $uv$ (if I didn't use $uv$ there would already be another cycle). So $\cycl - uv$ is a $u - v$ path in $T$.

    Suppose $P$ and $Q$ are two distinct $u - v$ paths in $T$, let $x$ be the first vertex on $P$ which is incident in $P$ to an edge not in $Q$. Let $y$ be the first vertex on $P$ after $x$ which is also in $Q$. Then $xPy$ and $yQx$ do not share any edge, so $xPyQx$ is a cycle in $T$.

    \boldmath$4 \implies 1$\unboldmath

    Suppose there exists a unique $u - v$ path between any two $u, v \in \v[T]$. Then, in particular, $T$ is connected. Moreover, if $T$ contained a cycle $\cycl$, any two vertices $u, v \in \v[\cycl]$ would be joined by two different paths.\qed
\end{prf}
\subsection{Spanning trees}
\begin{definition}
    The spanning tree for a connected graph $G$ is a subgraph $T \subseteq G$ which is a tree with $\v[T] = \v$
\end{definition}
\begin{customcorollary}{1.16}
    \label{corollary:1.16}
    Any connected graph $G$ has a spanning tree
\end{customcorollary}
\begin{prf}
    Let $H \subseteq G$ be a spanning connected subgraph of $G$ with as few edges as possible. This means that $H$ is minimally connected, thus, by Theorem \ref{theorem:1.15}, $H$ is a tree.
\end{prf}
\begin{customtheorem}{1.17}
    \label{theorem:1.17}
    A connected graph $T$ on $n$ vertices is a tree $\iff \ne[T] = n - 1$
\end{customtheorem}
\begin{prf}
    We will prove the above by induction on $n$. In the \textbf{Base case} we have $n = 1 \implies T = K_1$ and $\ne[T] = 0 = 1 - 1$.

    In the \textbf{inductive case} we have to prove the above in both directions.
    \begin{itemize}
        \item [($\implies$)] By Lemma \ref{lemma:1.14} $T$ has a leaf $l$ and so $T - l$ is a tree on $n - 1$ vertices. By induction hypothesis we have $\ne[T - l] = (n - 1) - 1$ thus $\ne[T] = n - 2 + 1 = n - 1$
        \item [($\impliedby$)] Because of Corollary \ref{corollary:1.16}, $T$ has a spanning tree $T'$ and we have shown that $\ne[T'] = n - 1$, thus
        \begin{equation*}
            \ne[T'] = \ne[T] \implies T = T'
        \end{equation*}
        That means that $T$ must be a tree.
    \end{itemize}
\end{prf}
\begin{coolfact}
    The key idea in this proof is that removing a leaf from a tree gives a tree on one less vertex, that proves very useful for inductions on the number of vertices.
\end{coolfact}
\begin{definition}[Root]
    A root is a special vertex inside the tree because it is not son of anyone, any tree possessing a root is called rooted tree.
\end{definition}
\begin{definition}[Reachability]
    Starting from a  vertex $v \in V$ we say that $w$ is reachable from $v$ if $w$ is inside $v$'s component.
\end{definition}
We will now see the first example of an algorithm in this course.
\subsection{\texttt{Graph Scanning algorithm}}
\subsubsection{Input}
A graph $G$ and a vertex $s \in \v$
\subsubsection{Output}
A spanning tree of the component containing $s$. More precisely, $R \subseteq \v$ and $T \subseteq \e$ such that $(R, T)$ is a spanning tree of the component containing $s$.
\subsubsection{Algorithm}
\begin{enumerate}
    \item Set $R = \grf{s}, Q = \grf{s}, T = \emptyset$
    \item \label{tag:condition} \texttt{if} $Q$ is empty \texttt{then} stop
    \begin{enumerate}
        \item \texttt{else} $v \in Q$
    \end{enumerate}
    \item \texttt{if} possible choose $w \in \v \setminus R$ such that $vw \in \e$
    \begin{enumerate}
        \item \texttt{if} there is no such $w$ let $Q = Q \setminus \grf{v}$ and go to \ref{tag:condition}
    \end{enumerate}
    \item Set $R = R \cup \grf{w}, Q = Q \cup \grf{w}$ and $T = T \cup \grf{w}$ and go to \ref{tag:condition}
\end{enumerate}
\begin{figure}
    \centering
    \begin{subfigure}[t]{0.45\textwidth}
        \begin{tikzpicture}
    \node[defv] (n1) at (-1,-2) {$v_1$};
    \node[defv] (n2) at (-1,0) {$v_2$};
    \node[defv] (n3) at (0,-1) {$v_3$};
    \node[defv] (n4) at (1,-2) {$v_4$};
    \node[defv] (n5) at (3,-1) {$v_5$};
    \node[defv] (n6) at (2,0) {$v_6$};
    \node[defv] (n7) at (2,1) {$v_7$};
    \node[defv] (n8) at (1,2) {$s$};

    \draw[defe] (n1) edge node[above] {} (n4);
    \draw[defe] (n1) edge node[above] {} (n2);
    \draw[defe] (n2) edge node[above] {} (n8);
    \draw[defe] (n2) edge node[above] {} (n7);
    \draw[defe] (n2) edge node[above] {} (n3);
    \draw[defe] (n3) edge node[above] {} (n6);
    \draw[defe] (n3) edge node[above] {} (n4);
    \draw[defe] (n4) edge node[above] {} (n5);
    \draw[defe] (n5) edge node[above] {} (n6);
    \draw[defe] (n6) edge node[above] {} (n7);

    \draw[ultra thick, rede, dashed] (n8) edge node[above] {} (n2);
    \draw[ultra thick, rede, dashed] (n2) edge node[above] {} (n7);
    \draw[ultra thick, rede, dashed] (n2) edge node[above] {} (n3);
    \draw[ultra thick, rede, dashed] (n2) edge node[above] {} (n1);
    \draw[ultra thick, rede, dashed] (n1) edge node[above] {} (n4);
    \draw[ultra thick, rede, dashed] (n3) edge node[above] {} (n6);
    \draw[ultra thick, rede, dashed] (n4) edge node[above] {} (n5);
\end{tikzpicture}
        \caption{Result of the BFS visit strategy}
    \end{subfigure}
    \hfill
    \begin{subfigure}[t]{0.45\textwidth}
        \begin{tikzpicture}
    \node[defv] (n1) at (-1,-2) {$v_1$};
    \node[defv] (n2) at (-1,0) {$v_2$};
    \node[defv] (n3) at (0,-1) {$v_3$};
    \node[defv] (n4) at (1,-2) {$v_4$};
    \node[defv] (n5) at (3,-1) {$v_5$};
    \node[defv] (n6) at (2,0) {$v_6$};
    \node[defv] (n7) at (2,1) {$v_7$};
    \node[defv] (n8) at (1,2) {$s$};

    \draw[defe] (n1) edge node[above] {} (n4);
    \draw[defe] (n1) edge node[above] {} (n2);
    \draw[defe] (n2) edge node[above] {} (n8);
    \draw[defe] (n2) edge node[above] {} (n7);
    \draw[defe] (n2) edge node[above] {} (n3);
    \draw[defe] (n3) edge node[above] {} (n6);
    \draw[defe] (n3) edge node[above] {} (n4);
    \draw[defe] (n4) edge node[above] {} (n5);
    \draw[defe] (n5) edge node[above] {} (n6);
    \draw[defe] (n6) edge node[above] {} (n7);

    \draw[ultra thick, blue, dashed] (n8) edge node[above] {} (n2);
    \draw[ultra thick, blue, dashed] (n2) edge node[above] {} (n1);
    \draw[ultra thick, blue, dashed] (n1) edge node[above] {} (n4);
    \draw[ultra thick, blue, dashed] (n4) edge node[above] {} (n3);
    \draw[ultra thick, blue, dashed] (n4) edge node[above] {} (n5);
    \draw[ultra thick, blue, dashed] (n3) edge node[above] {} (n6);
    \draw[ultra thick, blue, dashed] (n6) edge node[above] {} (n7);
\end{tikzpicture}
        \caption{Result of the DFS visit strategy}
    \end{subfigure}
    \label{fig:graph_scanning_algorithm}
\end{figure}
\begin{customlemma}{1.18}
    \label{lemma:1.18}
    The graph spanning algorithm works properly.
\end{customlemma}
\begin{prf}
    We have to show that the obtained graph is a tree and it must contain all of the vertices in the component.

    At any stage, $(R, T)$ is a tree (at the beginning it's just a single vertex and with every step we add a leaf to the tree). Suppose for a contradiction that there exists $w \in \v \setminus R$ but $w$ is reachable from $s$.

    Let $P$ be an $s - w$ path in $G$. Since $s \in R$, there exists $xy \in \e[P]$ such that $x \in R$ but $y \notin R$. Since $x \in R$, it must have been added to $Q$ at some point in the algorithm (during either step $1$ or step $4$). The algorithm does not stop before $Q = \emptyset$ and $x$ can only be removed from $Q$ when $\ngbrs{x} \subseteq R$. That means that $y$ must have been added to $Q$ at some point, but whenever a vertex is added to $Q$, it is also added to $R$, which implies that $y \in R$, but that is a contradiction.

    Therefore all vertices that are reachable from $s$ are in $R$. \qed
\end{prf}
\begin{customlemma}{1.19}
    \label{lemma:1.19}
    Suppose the \texttt{Graph Scanning algorithm} is applied to a connected graph $G$ with vertex $s \in \v$ and we use \texttt{BFS}. Then for any vertex $v \in \v$ we have $dist_G(s, v) = dist_{(R, T)}(s, v)$. Thus the \texttt{BFS} tree preserves distances to the root.
\end{customlemma}
\begin{prf}
    Let $n = \nv$ and $v_1, \dots, v_n$ be the enumeration of the vertices in $G$ in the order in which they are added to $Q$.
    \begin{claim}
        \label{claim:1_lemma1.19}
        For $1 \leq i \leq j \leq n$ we have
        \begin{equation*}
            dist_G(s, v_i) \leq dist_G(s, v_j)
        \end{equation*}
    \end{claim}
    \begin{prf}
        Suppose for a contradiction that $\exists v_l \in \v$ with $dist_{(R, T)}(s, v_l) \neq dist_G(s, v_l)$ and suppose $dist_G(s, v_l)$ is minimum. Since (R, T) is a subgraph of $G$ we have
        \begin{equation*}
            dist_{(R, T)}(s, v_l) > dist_G(s, v_l)
        \end{equation*}
        Let $P$ be an $s - v_l$ path of minimum length in $G$ and denote by $v_k$ the neighbour of $v_l$ on $P$ then $dist_G(s, v_k) = dist_G(s, v_l) - 1$ so, by assumption, $dist_G(s, v_k) = dist_{(R, T)}(s, v_k)$.
        Thus $v_k \notin T$, otherwise we could concatenate it to an $s - v_k$ path of length $dist_G(s, v_k)$ in $(R, T)$ to get an $s - v_k$ path of length $dist_G(s, v_k) + 1 = dist_G(s, v_l)$ in $(R, T)$. \qed
    \end{prf}
    By \ref{claim:1_lemma1.19} we have $k < l$, but $v_kv_k \notin T$ means that $v_l$ was in $R$ when scanning a $v_k$ so $\exists v_mv_l \in T$ for some $m < k$. By \ref{claim:1_lemma1.19}
    \begin{equation*}
        dist_G(s, vm) \leq dist_G(s, v_k) = dist_G(s, v_l) - 1
    \end{equation*}
    But by minimality we have an $s - v_m$ path of length $dist_G(s, vm) \leq dist_G(s, v_l) - 1$ in $(R, T)$, thus, together with $v_mv_l$ we get an $s - v_l$ path of length $\leq dist_G(s, v_l)$ in $(R, T)$, which is absurd. \qed
\end{prf}
\begin{customlemma}{1.20}
    \label{lemma:1.20}
    Applying \texttt{DFS} to a connected graph $G$ and vertex $s \in \v$, we obtain a spanning tree $T$ such that for any $xy \in \e$ there exists a leaf $v_l \in \v[T]$ such that $x$ and $y$ lie on the unique $s - v_l$ path in $T$
\end{customlemma}
\begin{prf}
    Let $xy \in \e$ and say without loss of generality that $x$ was added to the queue $Q$ first.

    At any stage of the algorithm, $T$ is a tree, so when $x$ is added to $Q$, $T$ is a tree containing $s$ and $x$ but not $y$. So there is an $s - x$ path $P$ in $T$ not using $y$. Let $e_1, \dots, e_k$ be the sequence of edges from the first edge added after $x$ enters $Q$ to the last edge before $x$ is removed from $Q$. Note that $T' = \bigcup_{i \in [k]}e_i$ is a subtree of $T$ such that $y \in \v[T']$ since $y \in N(x)$ and we cannot delete $x$ before $y$ is added to $Q$.

    There is an $x - y$ path $P'$ in $T$, thus $P \cup P'$ is an $s - y$ path using $x$ and $y$. Note that $P \cup P'$ is indeed a path since $T$ is a tree. If $y$ is a leaf, we are done, otherwise we can keep extending $P \cup P'$ until we hit a leaf.\qed
\end{prf}
\subsection{\texttt{Minimum Spanning Tree problem}}
\begin{customlemma}{1.22}
    \label{lemma:1.22}
    Let $G$ be a connected graph and $c: \e \rightarrow \R$. Let $T$ be a spanning tree. Then $T$ is a minimum spanning tree $\iff xy \in \e$, there is no edge $e$ in the unique $x - y$ path in $T$ such that $c(e) > c(xy)$. 
\end{customlemma}
\begin{prf}
    To prove this theorem we will have to prove both directions.
    \begin{itemize}
        \item[$(\implies)$] Let $T$ be a spanning tree, let $xy \in \e$ and $P$ be the $x - y$ path in $T$, suppose for a contradiction that there is $e \in \e[P]$ with $c(e) > c(xy)$. Note that $T' = T + xy - e$ has a smaller weight and is still spanning because we are still covering all fo the vertices. We claimed that $T'$ is a tree, which means that, by Theorem \ref{theorem:1.15}, $T$ is maximally acyclic, i.e. $T + xy$ has a cycle containing $xy$.

        This cycle consists of $P + xy$. $P$ unique implies a unique cycle in $T + xy$, deleting $e$ removes this cycle, so $T'$ is a tree. But that is a contradiction, so there is no such $e$.
        \item[$(\impliedby)$] Suppose $T$ satisfies the property of the lemma, let $T'$ be a minimum spanning tree which shares as many edges as possible with $T$. We want to show that $T' = T$.
        
        Suppose not, so there exists $xy \in \e[T']\setminus\e[T]$, let $P$ be the $x - y$ path in $T$. By assumption each edge in $P$ has weight $\leq c(xy)$. Since $T'$ is a tree, $\exists e \in \e[P] \setminus \e[T']$ (otherwise $P + xy$ would be a cycle in $T'$). By the same arguments as above, $T' + e - xy$ is still a spanning tree but has weight no larger than $T'$. Since $c(e) \leq c(xy)$. But $T' + e - xy$ has one more edge in common with $T$, which is absurd.\qed
    \end{itemize}
\end{prf}
\subsection{\texttt{Kruskal's algorithm}}
\subsubsection{Input}

Connected graph $G$, weights $c: \e \rightarrow \R$

\subsubsection{Output}

Minimum spanning tree

\subsubsection{Algorithm}
\begin{enumerate}
    \item Label the edges of $G$ with $e_1, \dots, e_m$ such that
    \begin{equation*}
        c(e_1) \leq c(e_2) \leq \dots \leq c(e_m)
    \end{equation*}
    \item Set $T = (\v, \emptyset)$
    \item \texttt{for} $i = 1$ to $m$ \texttt{do}:
    \begin{enumerate}
        \item \texttt{if} $T + e_i$ does not contain a cycle, set $T = T + e_i$
    \end{enumerate}
\end{enumerate}
\begin{customtheorem}{1.23}
    \label{theorem:1.23}
    Kruskal's algorithm works properly and runs in $\O(nm)$ time
\end{customtheorem}
\begin{prf}
    Note that $T$ has no cycle and we check all edges, so $T$ is a spanning tree. If $xy$ is not added to $T$, then it means that when $xy$ was considered, $T$ already had an $x - y$ path and all edges on this path have weight $\leq c(xy)$. So by Lemma \ref{lemma:1.22} $T$ is a minimum spanning tree.
\end{prf}