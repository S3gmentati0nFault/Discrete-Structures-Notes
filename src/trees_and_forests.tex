\section{Trees and forests}
A forest is a graph without cycles and a tree is a connected graph without cycles.
\begin{lemma}
    \label{lemma:1.14}
    Any tree with at least $2$ vertices must have at least two leaves.
\end{lemma}
\begin{prf}
    Let $P$ be a largest path, by maximality of $P$ the extremities do not have neighbours outside the path $P$ and have just one neighbour inside the path, since there are no cycles we have $N(v_1) = \{v_1\}$ and $N(v_l) = \{v_{l - 1}\}$
\end{prf}
\begin{theorem}
    \label{theorem:1.15}
    Let $T$ be a graph, the following statements are equivalent:
    \begin{enumerate}
        \item $T$ is a tree
        \item $T$ is minimally connected, i.e. $T$ is connected but $T - e$ is disconnected $\fa e \in \e[T]$
        \item $T$ is maximally acyclic, i.e. $T$ has no cycle but for any non adjacent $x \neq y \in \v[T], T + xy$ has a cycle
        \item For any $u, v \in \v[T]$ there is a unique $u - v$ path in $T$
    \end{enumerate}
\end{theorem}
\begin{prf}
    \boldmath$1 \implies 2$\unboldmath

    Suppose $T$ is a tree, by definition $T$ is connected. Let $uv \in \e[T]$, if $T - uv$ is connected, then $T$ has a $u - v$ path not using $uv$. Now $P + uv$ would be a cycle, which is absurd because we have a tree.

    \boldmath$2 \implies 3$\unboldmath

    Let's suppose by contradiction $T$ is minimally connected but contains a cycle, if we take $u, v \in \v[\cycl]$ and $u \in N_{\cycl}(v)$ then we know $uv \in \e[\cycl]$ thus removing the edge from the set $E(\cycl)$ should disconnect the tree because it's minimally connected, but there is at least one pair of vertices $x, y \in \v[\cycl]$ (possibly $\{x, y\} = \{u, v\}$) such that:
    \begin{itemize}
        \item There is a $u - x$ path in $\cycl$ that doesn't contain $uv$
        \item There is a $x - y$ path in $\cycl$ that doesn't contain $uv$
        \item There is a $y - v$ path in $\cycl$ that doesn't contain $uv$
    \end{itemize}
    But since the connection relationship is transitive (by Lemma \ref{lemma:1.11}) then there is stilla $u - v$ path even after removing $uv$, that means that the graph was not actually disconnected by the removal of $uv$ therefore we conclude that a tree has to be maximally acyclic since it's also minimally connected

    \boldmath$3 \implies 4$\unboldmath

    Suppose $T$ is maximally acyclic, let $u, v \in \v[T]$. If $uv \in \e[T]$, then $uv$ is a $u - v$ path. Otherwise, by assumption, $T + uv$ has a cycle $\cycl$ using $uv$ (if I didn't use $uv$ there would already be another cycle). So $\cycl - uv$ is a $u - v$ path in $T$.

    Suppose $P$ and $Q$ are two distinct $u - v$ paths in $T$, let $x$ be the first vertex on $P$ which is incident in $P$ to an edge not in $Q$. Let $y$ be the first vertex on $P$ after $x$ which is also in $Q$. Then $xPy$ and $yQx$ do not share any edge, so $xPyQx$ is a cycle in $T$.

    \boldmath$4 \implies 1$\unboldmath

    Suppose there exists a unique $u - v$ path between any two $u, v \in \v[T]$. Then, in particular, $T$ is connected. Moreover, if $T$ contained a cycle $\cycl$, any two vertices $u, v \in \v[\cycl]$ would be joined by two different paths.\qed
\end{prf}
\section{Spanning trees}
\begin{definition}
    The spanning tree for a connected graph $G$ is a subgraph $T \subseteq G$ which is a tree with $\v[T] = \v$
\end{definition}
\begin{corollary}
    \label{corollary:1.16}
    Any connected graph $G$ has a spanning tree
\end{corollary}
\begin{prf}
    Let $H \subseteq G$ be a spanning connected subgraph of $G$ with as few edges as possible. This means that $H$ is minimally connected, thus, by Theorem \ref{theorem:1.15}, $H$ is a tree.
\end{prf}
\begin{lemma}
    A connected graph $T$ on $n$ vertices is a tree $\iff \ne[T] = n - 1$
\end{lemma}
\begin{prf}
    We will prove the above by induction on $n$. In the \textbf{Base case} we have $n = 1 \implies T = K_1$ and $\ne[T] = 0 = 1 - 1$.

    In the \textbf{inductive case} we have to prove the above in both directions.
    \begin{itemize}
        \item [($\implies$)] By Lemma \ref{lemma:1.14} $T$ has a leaf $l$ and so $T - l$ is a tree on $n - 1$ vertices. By induction hypothesis we have $\ne[T - l] = (n - 1) - 1$ thus $\ne[T] = n - 2 + 1 = n - 1$
        \item [($\impliedby$)] Because of Corollary \ref{corollary:1.16}, $T$ has a spanning tree $T'$ and we have shown that $\ne[T'] = n - 1$, thus
        \begin{equation*}
            \ne[T'] = \ne[T] \implies T = T'
        \end{equation*}
        That means that $T$ must be a tree.
    \end{itemize}
\end{prf}
\begin{coolfact}
    The key odea in this proof is that removing a leaf from a tree gives a tree on one less vertex, that proves very useful for inductions on the number of vertices.
\end{coolfact}
\begin{definition}[Root]
    A root is a special vertex inside the tree because it is not son of anyone, any tree possessing a root is called rooted tree.
\end{definition}
\begin{definition}[Reachability]
    Starting froma  vertex $v \in V$ we say that $w$ is reachable from $v$ if $w$ is inside $v$'s component.
\end{definition}
We will now see the first example of an algorithm in this course.
\subsection{\texttt{Graph Scanning algorithm}}
\subsubsection{Input}
A graph $G$ and a vertex $s \in \v$
\subsubsection{Output}
A spanning tree of the component containing $s$. More precisely, $R \subseteq \v$ and $T \subseteq \e$ such that $(R, T)$ is a spanning tree of the component containing $s$ (assuming you could have a forest).
\subsubsection{Algorithm}
\begin{enumerate}
    \item Set $R = \{s\}, Q = \{s\}, T = \emptyset$
    \item \label{tag:condition} If $Q$ is empty then stop
    \begin{enumerate}
        \item Else $v \in Q$
    \end{enumerate}
    \item If possible choose $w \in \v \setminus R$ such that $vw \in \e$
    \begin{enumerate}
        \item If there is no such $w$ let $Q = Q \setminus \{v\}$ and go to \ref{tag:condition}
    \end{enumerate}
    \item Set $R = R \cup \{w\}, Q = Q \cup \{w\}$ and $T = T \cup \{w\}$ and go to \ref{tag:condition}
\end{enumerate}
\begin{lemma}
    The graph spanning algorithm works properly.
\end{lemma}
\begin{prf}
    We have to show that the obtained graph is a tree and it must contain all of the vertices in the component.

    At any stage, $(R, T)$ is a tree (at the beginning it's just a single vertex and with every step we add a leaf to the tree). Suppose for a contradiction that there exists $w \in \v \setminus R$ but $w$ is reachable from $s$.

    Let $P$ be an $s - w$ path in $G$. Since $s \in R$, there exists $xy \in \e[P]$ such that $x \in R$ but $y \notin R$. Since $x \in R$, it must have been added to $Q$ at some point in the algorithm (during either step $1$ or step $4$). The algorithm does not stop before $Q = \emptyset$ and $x$ can only be removed from $Q$ when $\ngbrs{x} \subseteq R$. That means that $y$ must have been added to $Q$ at some point, but whenever a vertex is added to $Q$, it is also added to $R$, which implies that $y \in R$, but that is a contradiction.

    Therefore all vertices that are reachable from $s$ are in $R$. \qed
\end{prf}
\begin{lemma}
    Suppose the \texttt{Graph Scanning algorithm} is applied to a connected graph $G$ with vertex $s \in \v$ and we use \texttt{BFS}. Then for any vertex $v \in \v$ we have $dist_G(s, v) = dist_{(R, T)}(s, v)$. Thus the \texttt{BFS} tree preserves distances to the root.
\end{lemma}
\begin{prf}
    Let $n = \nv$ and $v_1, \dots, v_n$ be the enumeration of the vertices in $G$ in the order in which they are added to $Q$.
    \begin{claim}
        \label{claim:1_lemma1.19}
        For $1 \leq i \leq j \leq n$ we have
        \begin{equation*}
            dist_G(s, v_i) \leq dist_G(s, v_j)
        \end{equation*}
    \end{claim}
    \begin{prf}
        Suppose for a contradiction that $\exists v_l \in \v$ with $dist_{(R, T)}(s, v_l) \neq dist_G(s, v_l)$ and suppose $dist_G(s, v_l)$ is minimum. Since (R, T) is a subgraph of $G$ we have
        \begin{equation*}
            dist_{(R, T)}(s, v_l) > dist_G(s, v_l)
        \end{equation*}
        Let $P$ be an $s - v_l$ path of minimum length in $G$ and denote by $v_k$ the neighbour of $v_l$ on $P$ then $dist_G(s, v_k) = dist_G(s, v_l) - 1$ so, by assumption, $dist_G(s, v_k) = dist_{(R, T)}(s, v_k)$.
        Thus $v_k \notin T$, otherwise we could concatenate it to an $s - v_k$ path of length $dist_G(s, v_k)$ in $(R, T)$ to get an $s - v_k$ path of length $dist_G(s, v_k) + 1 = dist_G(s, v_l)$ in $(R, T)$. \qed
    \end{prf}
    By \ref{claim:1_lemma1.19} we have $k < l$, but $v_kv_k \notin T$ means that $v_l$ was in $R$ when scanning a $v_k$ so $\exists v_mv_l \in T$ for some $m < k$. By \ref{claim:1_lemma1.19}
    \begin{equation*}
        dist_G(s, vm) \leq dist_G(s, v_k) = dist_G(s, v_l) - 1
    \end{equation*}
    But by minimality we have an $s - v_m$ path of length $dist_G(s, vm) \leq dist_G(s, v_l) - 1$ in $(R, T)$, thus, together with $v_mv_l$ we get an $s - v_l$ path of length $\leq dist_G(s, v_l)$ in $(R, T)$, which is absurd. \qed
\end{prf}
\begin{lemma}
    Applying \texttt{DFS} to a connected graph $G$ and vertex $s \in \v$, we obtain a spanning tree $T$ such that for any $xy \in \e$ there exists a leaf $v_l \in \v[T]$ such that $x$ and $y$ lie on the unique $s - v_l$ path in $T$
\end{lemma}
\begin{prf}
    Let $xy \in \e$ and say without loss of generality that $x$ was added to the queue $Q$ first.

    At any stage of the algorithm, $T$ is a tree, so when $x$ is added to $Q$, $T$ is a tree containing $s$ and $x$ but not $y$. So there is an $s - x$ path $P$ in $T$ not using $y$. Let $e_1, \dots, e_k$ be the sequence of edges from the first edge added after $x$ enters $Q$ to the last edge before $x$ is removed from $Q$. Note that $T' = \bigcup_{i \in [k]}e_i$ is a subtree of $T$ such that $y \in \v[T']$ since $y \in N(x)$ and we cannot delete $x$ before $y$ is added to $Q$.

    There is an $x - y$ path $P'$ in $T$, thus $P \cup P'$ is an $s - y$ path using $x$ and $y$. Note that $P \cup P'$ is indeed a path since $T$ is a tree. If $y$ is a leaf, we are done, otherwise we can keep extending $P \cup P'$ until we hit a leaf.\qed
\end{prf}
\section{\texttt{Minimum Spanning Tree problem}}
\begin{lemma}
    \label{lemma:1.22}
    Let $G$ be a connected graph and $c: \e \rightarrow \R$. Let $T$ be a spanning tree. Then $T$ is a minimum spanning tree $iff xy \in \e$, there is no edge $e$ in the unique $x - y$ path in $T$ such that $c(e) > c(xy)$. 
\end{lemma}
\begin{prf}
    \begin{itemize}
        \item[$(\implies)$] Let $T$ be a spanning tree, let $xy \in \e$ and $P$ be the $x - y$ path in $T$, suppose for a contradiction that there is $e \in \e[P]$ with $c(e) > c(xy)$. Note that $T' = T + xy - e$ has a smaller weight and is still spanning because we are still covering all fo the vertices. We claimed that $T'$ is a tree, which means that, by Theorem \ref{theorem:1.15}, $T$ is maximally acyclic, i.e. $T + xy$ has a cycle containing $xy$.

        This cycle consists of $P + xy$. $P$ unique implies a unique cycle in $T + xy$, deleting $e$ removes this cycle, so $T'$ is a tree. But that is a contradiction, so there is no such $e$.
        \item[$(\implies)$] Suppose $T$ satisfies the property of the lemma, let $T'$ be a minimum spanning tree which shares as many edges as possible with $T$. We want to show that $T' = T$.
        
        Suppose not, so there exists $xy \in \e[T']\setminus\e[T]$, let $P$ be the $x - y$ path in $T$. By assumption each edge in $P$ has weight $\leq c(xy)$. Since $T'$ is a tree, $\exists e \in \e[P] \setminus \e[T']$ (otherwise $P + xy$ would be a cycle in $T'$). By the same arguments as above, $T' + e - xy$ is still a spanning tree but has weight no larger than $T'$. Since $c(e) \leq c(xy)$. But $T' + e - xy$ has one more edge in common with $T$, which is absurd.\qed
    \end{itemize}
\end{prf}
\subsection{\texttt{Kruskal's algorithm}}
\subsubsection{Input}
Connected graph $G$, weights $c: \e \rightarrow \R$
\subsubsection{Output}
Minimum spanning tree
\subsubsection{Algorithm}
\begin{enumerate}
    \item Label the edges of $G$ with $e_1, \dots, e_m$ such that
    \begin{equation*}
        c(e_1) \leq c(e_2) \leq \dots \leq c(e_m)
    \end{equation*}
    \item Set $T = (\v, \emptyset)$
    \item For $i = 1$ to $m$ do:
    \begin{enumerate}
        \item if $T + e_i$ does not contain a cycle, set $T = T + e_i$
    \end{enumerate}
\end{enumerate}
\begin{theorem}
    Kruskal's algorithm works properly and runs in $\O(nm)$ time
\end{theorem}
\begin{prf}
    Note that $T$ has no cycle and we check all edges, so $T$ is a spanning tree. If $xy$ is not added to $T$, then it means that when $xy$ was considered, $T$ already had an $x - y$ path and all edges on this path have weight $\leq c(xy)$. So by Lemma \ref{lemma:1.22} $T$ is a minimum spanning tree.
\end{prf}
