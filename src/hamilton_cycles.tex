\chapter{Hamilton cycles}
\begin{definition}[Hamilton cycle]
    Let $G$ be a graph, an Hamilton cycle in $G$ is a spanning cycle and an Hamilton path is a spanning path.
\end{definition}
We say $G$ is Hamiltonian if it contains an Hamilton cycle. The following is the sufficient condition for hamiltonicity.
\begin{customtheorem}{Dirac}
\label{theorem:dirac}
    Let $G$ be a graph on $n \geq 3$ vertices, with $\mind \geq \frac{n}{2}$. Then $G$ is Hamiltonian.
\end{customtheorem}
\begin{prf}
    Note $G$ is connected, since otherwise $G$ woudl have a component of size $\leq \frac{n}{2}$ and the vertices in that component would have degree $\leq \frac{n}{2} - 1$.

    Let $P = v_0v_1\dots v_l$ be a maximum path in $G$, by maximality, we know that $\ngbrs{v_0}, \ngbrs{v_l} \subseteq \v[P]$. If $v_0v_l \in \e$ then $P + v_0v_l$ is a cycle $\cycl$ in $G$ with $\v[\cycl] = \v[P]$. Otherwise, let $I \subseteq \{1, \dots, l - 1\}$ be the set of $i$ such that $v_lv_i \in \e, |I| \geq \mind = \frac{n}{2}$.

    Recall $\ngbrs{v_0} \subseteq \{v_1, \dots, v_{l - 1}\}$ where $|\{v_1, \dots, v_{l - 1}\}| \leq n - 2$. So $v_0$ has $< n - 2 - \frac{n}{2} = \frac{n}{2} - 2$ non neighbours in $\{v_1, \dots, v_{l - 1}\}$. Thus there is $i \in I$ such that $v_{i + 1} \in \ngbrs{v_0}$, thus we have a cycle $\cycl = v_0v_1 \dots v_iv_lv_{l - 1} \dots v_{i + 1}v_0$ in $G$ with $\v[\cycl] = \v[P]$.

    Since $G$ is connected, then if $\v[\cycl] \neq \v[G]$ then we have a vertex $v \in \v \setminus \v[\cycl]$ with a neighbour on $\cycl$. Let $w'$ be a neighbour of $w$ on $\cycl$. Now $vw\cycl w'$ is a path in $G$ with one more vertex than $P$. This contradicts the maximality of $P$ so $\v[\cycl] = \v$. This means that $\cycl$ is an Hamilton cycle.\qed
\end{prf}
Recall that an independent set in $G$ is a set of pairwise non adjacent vertices. The independence number $\ind$ is the maximum size of an independent set in $G$ (we recall that $K(G)$ is the connectivity).
\begin{customtheorem}{5.2}
\label{theorem:5.2}
    If we have a graph $G$ on $\geq 3$ vertices, if $\ind \leq K(G)$ then $G$ is Hamiltonian.
\end{customtheorem}
\begin{definition}[Degree sequence]
    Let $G$ be a graph on $n$ vertices with degrees $d_1 \leq d_2 \leq \dots \leq d_n$ then the $n$-tuple $d_1 \dots d_n$ is the degree sequence of $G$.
\end{definition}
An $n$-tuple $d_1 \dots d_n$ with $d_1 \leq \dots \leq d_n$ is Hamiltonian if any graph with degree sequnce pointwise greater than $d_1 \dots d_n$ is Hamiltonian. $(a_1, \dots, a_n)$ Hamiltonian if all $n$-vertex $G$ with degree sequence pointwise greater than $(a_1, \dots, a_n)$ is Hamiltonian.
\begin{customproposition}{5.2}
\label{proposition:5.2}
    Any $G$ on $n \geq 3$ vertices with $\ind \leq k(G)$ is Hamiltonian.
\end{customproposition}
\begin{prf}
    Suppose not and let $\cycl$ be a cycle of max length in $G$ and let $u \in \v \setminus \v[\cycl]$ let $F$ be a maximum $U - \cycl$ fan.

    Note that $\cycl$ exists since $n \geq 3$ implies either any $3$ vertices form a cycle or there is an independent set of size $2$ which means that $k(G) \geq \ind \geq 2$ in which case we also have a cycle. Denote $C = v_1v_2 \dots v_l$ and let $I = \{i \in [l]: F$ contains a $u-v_i$ path $\}$.

    By \ref{theorem:menger}'s theorem the number of paths in $F$ is
    \begin{equation}
        \label{equation:1_5.2}
        |I| \geq min\{k(G), |\cycl|\}
    \end{equation}
    if there is $i \in [l]$ such that both $i, i + 1 \in I$ then we obtain a cycle $v_{i + 1} \cycl v_i F u F v_{i + 1}$ which contradicts the maximality of $\cycl$ thus $|I| < |\cycl|$ thus \ref{equation:1_5.2} implies $|I| \geq k(G)$.

    If $i, j \in I$ such that $v_{i + 1}v_{j + 1} \in \e$, then we obtain a new cycle $v_{j + 1} \cycl v_i F u F v_j \cycl v_{i + 1}$ which contradicts the maximality of $\cycl$. So $\{i + 1: i \in I\}$ is an independent set of size $|I| \geq k(G) \geq \ind$ of size $I \geq k(G) \geq \alpha(G)$ which means that $|J| = \alpha(G)$.

    But note $J \cap I = \emptyset$ by the above, then, by maximality of $F$, we have $uv_{i + 1} \notin \e$ for each $i + 1 \in J$ then $J \cup \{u\}$ is an independent set of size $\alpha(G) + 1$.\qed
\end{prf}
\begin{customtheorem}{Chvátal}
\label{theorem:chvatal}
    Let $(a_1, \dots, a_n)$ be an integer sequence with $0 \leq a_1 \leq \dots \leq a_n \leq n - 1$ and $n \geq 3$ then $(a_1, \dots, a_n)$ is Hamiltonian if and only if
    \begin{equation}
        \label{equation:1_chvatal}
        \fa i < \frac{n}{2} \st a_n \leq i \implies a_{n - i} \geq n - i
    \end{equation}
\end{customtheorem}
Note that \ref{theorem:dirac}'s theorem gives that $(\cel{\frac{n}{2}}, \dots, \cel{\frac{n}{2}})$ is Hamiltonian.
\begin{prf}
    To prove this theorem we have to prove the two different directions separately.
    \begin{itemize}
        \item[($\impliedby$)] Let $(a_1, \dots, a_n)$ be an integer sequence as in the statement satisfying \ref{equation:1_chvatal}, let's suppose for a contradiction it's not Hamiltonian. This means that there is an $n$-vertex $G$ with degree sequence $(d_1, \dots, d_n)$ such that $d_i \geq a_i \fa i \in [n]$ and $G$ has no Hamiltonian cycle. We take such a $G$ with $\ne$ maximum. Note $(d_1, \dots, d_n)$ satisfies \ref{equation:1_chvatal}.
        
        Let $x \neq y \in \v$ be non adjacent vertices with $d(x) \leq d(y)$ and $d(x) + d(y)$ is maximum. We can find these two vertices because if we couldn't we would have a complete graph and thus an Hamilton cycle, by \ref{theorem:dirac}'s theorem. Note that the degree sequence of $G + xy$ is pointwise greater than $(d_1, \dots, d_n)$ and so than $(a_1, \dots, a_n)$ then by maximality of $\ne$ implies that $G + xy$ contains an Hamilton cycle $v_1v_2 \dots v_n$ where $v_1 = x$ and $v_n = y$.

        Note that $v_1v_2 \dots v_1$  is a Hamilton path in $G$. let $I = \{i \in [n]: xv_{i + 1} \in \e\}$ and $J = \{i \in [n]: yv_i \in \e\}$. Same argument as \ref{theorem:dirac}'s proof: $\fa i \in J, i \notin I$ otherwise $v_1 v_{i + 1} v_{i + 2} \dots v_n v_i v_{i - 1} \dots v_1$ is a Hamilton cycle in $G$ (not using $xy = v_1v_n$ anymore) and that is absurd.

        From that comes that $I \cap J = \emptyset \implies |I \cup J| = d(x) + d(y) \leq n - 1$, because $I \subseteq \{1, \dots, n - 2\}$ and $J \subseteq \{2, \dots, n - 1\}$.

        We will now define an $h$ value as follows:
        \begin{equation*}
            h \das d(x) \leq \frac{n - 1}{2} < \frac{n}{2}
        \end{equation*}
        We showed that $v_i \in \ngbrs{y} \fa i \in I$, id est $v_i$ and $y$ are not adjacent, so the maximality of $d(x) + d(y)$ implies that $d(v_i) \leq d(x) = h$.

        In other words, $I$ is a set of $|I| = d(x) = h$ indices $i$ such that $d(v_i) \leq h$ which means that there are $h + 1$ vertices of degree $\leq h \implies x $ and the ones indexed by $I$. So $d_h \leq h \implies$ \ref{equation:1_chvatal} implies that $d_{n - h} \geq n - h$, this means that we have a degree sequence like this:
        \begin{equation*}
            (\underbrace{\leq h, \leq h, \dots, \leq h, \leq h}_\text{$h + 1$ times}, \dots, \underbrace{\geq n - h, \geq n - h, \dots, \geq n - h, \geq n - h}_\text{$h + 1$ times})
        \end{equation*}
        Which means that there are $h + 1$ vertices with degree $\geq n - h$, since $d(x) = h, \ex z \in \v$ with $d(z) \geq n - h$ such that $xz \notin \e$, but $d(x) + d(z) \geq h + (n - h) = n > d(x) + d(y)$ contradicting the maximality of $d(x) + d(y)$.
        \item[($\implies$)] We need to show that for any $(a_1, \dots, a_n)$ not satisfying \ref{equation:1_chvatal} there exists a graph $G$ with degree sequence pointwise greater than $(a_1, \dots a_n)$ which has no Hamiltonian cycle. Let $(a_1, \dots, a_n)$ as ub tge statenebt not satisfying \ref{equation:1_chvatal}, this means there is $h < \frac{n}{2}$ such that $a_h \leq h$ and $a_{n - h} < n - h$ the degree sequence is.
        \begin{equation*}
            (
                \underbrace{h, \dots, h}_{h},
                \underbrace{n - h - 1, \dots, n - h - 1}_{n - 2h},
                \underbrace{n - 1, \dots, n - 1}_{h}
            )
        \end{equation*}
        Thus it's pointwise greater than $(a_1, \dots, a_n)$ but $G$ has no Hamilton cycle, because any Hamilton cycle $\cycl$ needs to contain two edges incident to each $v_i$ for $i \in [h]$ and these lie in the $\clq{h,h} \implies$ any cycle in $G$ covering $\{v_1, \dots, v_h\}$ is a cycle on $\{v_1, \dots, v_{2h}\}$. Since $h < \frac{n}{2}$ such a cycle is not Hamiltonian.\qed
    \end{itemize}
\end{prf}
\section{Travelling salesman problem}
\textbf{Instance}

A $\clq{n}$ and edge weights $c: \e[\clq{n}] \rightarrow \R$

\noindent\textbf{Task}

A Hamilton cycle $H$ with minimum weights.

\noindent TSP is still NP-hard even under some restrictions / semplifications thus we need to accept the fact that our result will be approximated. A $k$-factor approximation for TSP is a polynomial time algorithm which outputs a Hamilton cycle of weight $\leq k \cdot min c(H)$. For TSP, there is no $k$-factor algorithm with $k \geq 1$.

\section{Metric TSP}
\textbf{Instance}

A $\clq{n}$ with edge-weights $c: \e \rightarrow \R$ satisfying the triangle inequality.

\noindent\textbf{Task}

Find a minimum weight Hamiltonian cycle $H$.

\begin{customlemma}{5.4}
\label{lemma:5.4}
    Given an instance $(\clq{n}, c)$ of a metric TSP and a connected spanning subgraph $G$ of $\clq{n}$, possibly with parallel edges, such that $\deg{v}$ is even for all $v \in \v$, then there exists an algorithm which outputs a Hamiltonian cycle $H$ of $\clq{n}$ with weight $\leq \sum_{e \in \e}c(e)$ and we can do this in linear time.
\end{customlemma}
\begin{prf}
    Since the degrees are even in the subgraph $G$ we can find an Euler tour $W$ and we can do that through Euler's algorithm in $\O(m)$ time, let $H$ be the Hamilton cycle of $\clq{n}$ whichi orders the vertices in the order they were first traversed in $W$.

    Note that for each $e = xy \in \e[H]$, we have a corresponding $x, y$-path, $P_e$ in $W$ and these are all edge-disjoint.

    The triangle inequality gives us:
    \begin{equation*}
        c(H) \leq \sum_{e \in \e[H]}c(P_e) \leq \sum_{e \in \e} c(e)
    \end{equation*}
    And that proves the theorem.\qed
\end{prf}
The best approximation algorithm for Metric TSP is Christofides' algorithm.
\section{\texttt{Christofides' algorithm}}
\subsection{Input}
    An instance $(\clq{n}, c)$ of metric TSP.
\subsection{Output}
    A Hamilton cycle of $\clq{n}$
\subsection{Algorithm}
\begin{enumerate}
    \item Find a minimum spanning tree $T$ of $\clq{n}$
    \item Let $W$ be the set of vertices of odd degree in $T$.
    \begin{enumerate}[label=\Alph*]
        \item Find a matching $J$ of minimum weight in $\clq{n}[W]$
        \item Let $G = (\v[\clq{n}], \e[H] \cup J)$. Let $T$ be a Hamilton cycle of $\clq{n}$ obtained by applying Lemma \ref{lemma:5.4}.
    \end{enumerate}
\end{enumerate}
\begin{customtheorem}{5.5}
\label{theorem:5.5}
    Christofides' algorithm is a $\frac{3}{2}$-factor approximation of Metric TSP and runs in $\O(n^3)$ time.
\end{customtheorem}
\begin{prf}
    For the running time we know that:
    \begin{enumerate}
        \item $\O(m + nlogn)$
        \item $\O(n^3)$
        \item By Lemma \ref{lemma:5.4} $\implies \O(m)$
    \end{enumerate}
    Overall $\O(n^3)$.

    Note that removing any edge from a Hamilton cycle gives a spanning tree, which means $c(H) \leq \text{minimum weight of Hamilton cycle in }\clq{n}$. Any Hamilton cycle in $\clq{n}[W]$ consists of two perfect matchings of $\clq{n}[W]$, so 
    \begin{align*}
        c(J) &\leq \text{minimum weight of Hamilton cycle of }\clq{n}[W]\\
             &\leq \frac{1}{2} \cdot \text{minimum weight of Hamilton cycle}
    \end{align*}
    therefore:
    \begin{equation*}
        c(H) \leq \frac{3}{2} \cdot \text{minimum weight of Hamilton cycle in } \clq{n}
    \end{equation*}
    and that proves our theorem.\qed
\end{prf}
