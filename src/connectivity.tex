\chapter{Connectivity}
\section{2-connectivity}
We recall that a graph is $2$-connected when it has at least $3$ vertices and the graph stays connected when we remove one vertex.
\begin{customproposition}{3.1}
\label{proposition:3.1}
    A graph $G$ is $2$-connected $\iff$ there is a sequence $\ears = G$ of $2$-connected subgraphs such that:
    \begin{itemize}
        \item $H_1$ is a cycle
        \item $\fa i \in [k - 1], H_{i + 1}$ is obtained from $H_i$ by adding a path whose endpoints are in $\v[H_i])$ but not its internal vertices.
    \end{itemize}
\end{customproposition}
\begin{definition}[Ears]
    A sequence $\ears$ is called an ear decomposition of $G$. The paths $\e[H_{i + 1}] \setminus \e[H_i]$ are called ears.
\end{definition}
More generally, given a graph $G$ and $H \subseteq G$ or $H \subseteq \v$, an $H$-path is a path in $G$ whose endpoints are in $H$ but whose internal vertices are in $G-H$.
\begin{prf}
    To prove Proposition \ref{proposition:3.1} we will prove first one direction, and then the other.
    \begin{itemize}
        \item [($\implies$)] Trivial as $H_n = G$ which is $2$-connected by assumption.
        \item [($\impliedby$)] Let $G$ be $2$-connected. Let $\ears$ be an ear decomposition for $G$ where $n$ is maximum. We want to show that $H_n = G$, with $n \geq 1$ since $G$ has a cycle.
        
        Suppose for a contradicition that $H_n \neq G$, suppose that there is $xy \in \e \setminus \e[H_n]$ with $x, y \in \v[H_n]$, then $xy$ is an $H_n$-path, thus we could add it to the ear decomposition and contradict the maximality of $k$. That means that $H_n$ is an induced subgraph of $G$, containing a subset of $\v$ but having all of the edges linking them, thus $H_n \neq G \implies \v \setminus \v[H_n] \neq \emptyset$.

        Since $G$ is connected, there is a vertex $v \in \v \setminus \v[H_n]$ which has a neighbor $u \in \v[H_n]$, since $G$ is $2$-connected $G - u$ is still a connected graph and thus there is a path $P$ from $v$ to $\v[H_n]$ in $G - u$. By taking such a minimum path we may assume that the internal vertices of $P$ lie outside of $H_n$. Note that $u \notin \v[P]$ so $uPv$ is an $H_n$-path and that could be added to the ear decomposition, contradicting the maximality of $n$.\qed
    \end{itemize}
\end{prf}
\begin{definition}[Cut vertex]
    A cut vertex is a vertex in a connected graph whose removal increases the number of connected components.
\end{definition}
\begin{definition}[Block]
    A block is a maximally connected subgraph with no cut vertices.
\end{definition}
The easiest example of a block is a single edge.
\begin{customlemma}{3.2}
\label{lemma:3.2}
    Let $G$ be a graph, then the following holds:
    \begin{enumerate}
        \item Every cycle is completely contained in a block of $G$
        \item Two distinct blocks meet at most in one vertex which will then be a cut vertex for $G$
        \item $\e$ is partitioned between the blocks
    \end{enumerate}
\end{customlemma}
\begin{prf}
    To prove the above statement we will have to prove the singulare statements separately:
    \begin{enumerate}
        \item Suppose $\cycl$ is some cycle in $G$ not completely contained in a block $B$. Pick an edge of $\cycl$, say $e \in B$. Go along $\cycl$ starting from $e$, then there is some edge that does not belong to $B$. The path starting with the first edge outside of $B$ is the start of some $B$-path, which is a contradiction because that makes $B$ not a block.
        \item Suppose we have $B$ and $B'$ two blocks, assume for a contradiction that $B, B'$ meet in $v$ and $v$ is not a cut vertex, then there is a $B - B'$ path with end vertices $x, y$ then there is an $x - y$ path in $B \cup B'$ therefore there is a cycle not completely contained in a block, which contradicts point 1.
        \item Follows from points 1. and 2.\qed
    \end{enumerate}
\end{prf}
For any graph $G$ we define the bipartite graph called the block tree on $X \cup \mathcal{B}$, where $X$ is the set of all cut vertices and $\mathcal{B}$ is the set of all blocks, xB will be an edge in the block tree if $x \in \v[B]$
\begin{customproposition}{3.3}
\label{proposition:3.3}
    The block tree of a connected graph is a tree.
\end{customproposition}
\section{3-connectivity}
\begin{customlemma}{3.4}
\label{lemma:3.4}
    Every $3$-connected graph $G$ on at least $5$ vertices has an edge $e$ such that $G / e$ is still $3$-connected.
\end{customlemma}
We call $G / e$ "G contracted e" and $/$ is the contraction operator.
\begin{prf}
    Suppose the statement is false, that means that $G / e$ is not $3$-connected $\fa e \in \e$. In particular there must be a $z$ such that $V_{xy}z$ is a separator in $G / xy$ which means that $x, y, z$ is a separator for the original expanded graph $G$.

    Pick $xy, z$ and a component $C$ of $G - \{x, y, z\}$ such that $C$ is of minimal order. Pick $v$, which is a neighbour of $z$ in $C$. Let $w$ be a vertex such that $v, z, w$ is a separator in $G$. Since $xy$ is an edge, there is a component $D$ in $G - \{v, z, w\}$ not containing $x, y$. We claim $D \subseteq C$, which is a contradiction because $D \subseteq C - v$.

    As $x, y, z$ are not contained in $D$ it suffices to show that $C \cap D \neq \emptyset$. Let $u$ be a neighbour of $v$ in $D$, since $u \notin \{x, y, z\}$ and since $u \in \v[D]$, the vertex $u$ lies also in $C$, which is a contradiction.\qed
\end{prf}
\begin{customtheorem}{3.5}
\label{theorem:3.5}
    A graph $G$ is $3$-connected if and only if there is a sequence $K_h = G_0, \dots, G_h = G$ where $G_{i + 1}$ contains an edge $xy$ with $\deg[G_{i + 1}]{y} \geq 3$ such that $G_i = G_{i + 1} / xy \fa i$. Moreover, each graph in $G_0, \dots, G_h$ is $3$-connected.
\end{customtheorem}
\begin{prf}
    Given $G$, by Lemma \ref{lemma:3.4} we have such a sequence.

    Assume we have such a sequence given, by induction over $h$ we show that $G$ is $3$-connected. Suppose there is a seprator $X$ in $G_h$ of size at most $2$, if $X$ separates two vertices, none of which is $x$ or $y$ then $X$ or $X \setminus \{x, y\} \cup \{V_{xy}\}$ is a separator of size at most $2$ in $G_{h - 1}$. Thus $G_h - X$ contains a component $C$, with $\v[C] \subseteq \{x, y\}$. Since $\deg[G_h]{X}, \deg[G_h]{y} \geq 3$ this is impossible.\qed
\end{prf}
\begin{definition}[$A - B$ path]
    Let $A, B \subseteq \v$ we define $A - B$ path a path in $G$ such that the first vertex of the path is in $A$, the last vertex of the path is in $B$ and all of the internal vertices are neither in $A$ or in $B$.    
\end{definition}
\begin{definition}[$A - B$ separator]
    Let $A, B$ be two blocks, let $X \subseteq \v$ such that $\v[X] \cap (\v[A] \cup \v[B]) = \emptyset$. $X$ is called $A - B$ separator if $X$'s  removal breaks $A - B$ connectivity.
\end{definition}
\begin{customtheorem}{Menger}
\label{theorem:menger}
    Let $A, B \subseteq \v$ be two vertex sets in $G$, then there are $k$ vertex disjoint $A - B$ paths in $G$ if and only if there is no $A - B$ separator of size $< k$.
\end{customtheorem}
\begin{prf}
    Let $P$ and $Q$ be two sets of disjoint $(A - B)$ paths. $Q$ exceeds $P$ if the first vertices of $P$ are a proper subset of the first vertices of $Q$ and if the last vertices of $P$ are a proper subset of the last vertices of $Q$.

    Let $k(A, B)$ be the size of the smallest $A - B$ separator. Show by induction on the number of vertices $P$. If $P$ contains fewer paths than $k(A, B)$ paths, then there is a path system $Q$ that exceeds $P$ with $|Q| = |P| + 1$ and $k(A, B_Q) \geq k(A, B)$\qed
\end{prf}
\begin{definition}[$A - B$ fan]
    A family of $k$ internally disjoint $(A, B)$-paths whose terminal vertices are distinct is an $A - B$ fan.
\end{definition}
\begin{customcorollary}{3.7}
\label{corollary:3.7}
    Let $G$ be a graph, $a \in \v$ and $B \subseteq \v \setminus \{a\}$. Then $G$ contains an $a - B$ fan with $k$ paths if and only if there is no $A - B$ separator (not containing $a$).
\end{customcorollary}
\begin{customcorollary}{3.8}
\label{corollary:3.8}
    For any $2$ non adjacent vertices $a, b$ in a graph $G$. There are $k$ internally vertex disjoint $a - b$ paths if and only if there is no $a - b$ separator $X \subseteq \v \setminus \{a, b\}$ of size at most $k - 1$.
\end{customcorollary}
\begin{customcorollary}{3.9}
\label{corollary:3.9}
    A graph $G$ is $k$-connected if and only if it contains $k$ vertex-disjoint $a - b$ paths for all $a, b \in \v$.
\end{customcorollary}
\begin{prf}
    Assume $G$ is $k$-connected. Let $a, b \in \v$, if $ab \notin \e$ then we use Corollary \ref{corollary:3.8}. Assume that $ab \in \e$, consider $G' = G - ab$ that means that $G'$ is $(k - 1)$-connected, then, because of Corollary \ref{corollary:3.8}, we have $k - 1$ internally vertex disjoint $ab$ paths in $G'$, this means that we can add $ab$.\qed
\end{prf}
\begin{definition}[Line graph]
    $\v[L(G)] = \e$
    
    $e, f \in \v[L(G)]$ are adjacent in $L(G)$ if $e, f$ share a common vertex.
\end{definition}
\begin{customcorollary}{3.10}
\label{corollary:3.10}
    Let $A, B \subseteq \v$ be two disjoint vertex sets in a graph $G$, then there are $k$ edge disjoint $A - B$ paths in $G$ if and only if there is no set of at most $k - 1$ edges that separate $A$ from $B$.
\end{customcorollary}
\begin{customcorollary}{3.11}
\label{corollary:3.11}
    Let $G$ be a graph and $a, b \in \v$ be distinct. Then $G$ contains $k$ edge disjoint $a - b$ paths if and only if there is no set of edges $F$ such that $|F| \leq k - 1$ and $G - F$ has no $a - b$ path.
\end{customcorollary}
\begin{customcorollary}{3.12}
\label{corollary:3.12}
    A graph $G$ is $k$ edge connected if and only if there exist $k$ edge disjoint paths between any two vertices in $G$.
\end{customcorollary}
\section{Maximum flow minimum cut}
We want to build an algorithm that can compute disjoint $a - b$ paths. Suppose $G$ is a directed graph. In addition, we have capacities on the edges $u: \e \rightarrow \R$.

We have two special vertices: source $s$ and target $t$.

The quadruple $(G, u, s, t)$ is known as a network. Given a network, the flow is a function $f: \e \rightarrow \R^0_+$ if $f(e) \leq u(e) \fa e \in \e$. The excess of a flow $f$ at a vertex $v$ is:
\begin{equation*}
    ex_f(v) = \sum_{e \in d^-(v)}f(e) - \sum_{e \in d^+(v)}f(e)
\end{equation*}
where $d^-(v)$ refers to the edges entering $v$ and $d^+(v)$ refers to the edges exiting $v$. An $s - t$ flow is a flow $f$ such that $ex_f(s) \leq 0$ and $ex_f(v) = 0, \fa v \in \v \setminus \{s, t\}$, we define the value of $f$ by $- ex_f(s)$.\qed
\section{\texttt{Max flow problem}}
\subsection{Input}
Given a network $(G, u, s, t)$
\subsection{Output}
Find an $s - t$ flow of max value.

An $s - t$ cut in $G$ is an edge set $\delta^+(X)$ with $s \in X$ and $t \notin X$, where $\delta^+(X)$ refers to the set of edges leaving $X$. The capacity of a cut is the sum of all capacities of its elements, we refer to it as $c(S, T)$.
\begin{customlemma}{3.13}
\label{lemma:3.13}
    Given a network $(G, u, s, t)$, a set $A \subseteq \v$ with $s \in A, t \notin A$ and $s - t$ flow $f$, then:
    \begin{enumerate}
        \item\label{lemma:3.13_1} $Value(f) = \sum_{e \in \delta^+(A)}f(e) - \underbrace{\sum_{e \in \delta^-(A)}f(e)}_{\geq 0}$.
        \item\label{lemma:3.13_2} $Value(f) = \sum_{e \in \delta^+(A)}f(e)$.
    \end{enumerate}
\end{customlemma}
\begin{prf}
    \begin{align*}
        Value(f) &= \sum_{e \in \delta^+(s)}f(e) - \sum_{e \in \delta^-(s)}f(e)\\
        &=\sum_{v \in A}(\sum_{e \in \delta^+(v)}f(e) - \sum_{e \in \delta^-(v)}f(e))\\
        &=\sum_{e \in \delta^+(A)}f(e) - \sum_{e \in \delta^-(A)}f(e)
    \end{align*}
    The second point of Lemma \ref{lemma:3.13} follows from the first.\qed
\end{prf}
\begin{customtheorem}{Ford - Fulkerson}
\label{theorem:ford-fulkerson}
    In any network the maximum value of an $s - t$ flow equals the minimum capacity of an $s - t$ cut.
\end{customtheorem}
Now we are going through some more notation.

Given a directed graph $G$, let $\snet$ be obtained from $G$ by adding for each edge $e = (u, v)$, the edge $\cev$. Given a directed graph $G$ and a capacity function $u$, and a flow $f$, we define the residual capacities $u_f: \e[\snet]$ bu $u_f = u(e) - f(e)$ and $u_f(\cev) = f(e) \fa e \in \e$. The residual graph $G_f$ is the graph with vertex set $\v$ and edge set $\{e \in \e[\snet]\st u_f(e) > 0\}$.

Given a flow $f$ and a path or cycle $P$ in $G_f$, we say we augment $P$ by $\gamma$ if we perform the following operation for each $e \in \e[P]$ such that if $e \in \e$ then increase $f(e)$ by $\gamma$ and otherwise $e = \cev[e']$ we decrease $f(e')$ by $\gamma$.

In a network $(G, u, s, t)$ and $s - t$ flow $f$, an $f$-augmenting path is an $s - t$ path in the residual graph $G_f$.
\begin{customtheorem}{3.1}
\label{theorem:3.1}
    Let $(G, u, s, t)$ be a network and $f$ a flow. Then the following statements are equivalent:
    \begin{enumerate}
        \item $Value(f)$ is max.
        \item $G$ has no $f$-augmenting path.
        \item There is an $s - t$ cut $(S, T)$ with $Value(f) = c(S, T)$.
    \end{enumerate}
\end{customtheorem}
\begin{prf}
    To prove the theorem we will prove all the possible combinations, to show that the statements are equivalent.
    \begin{itemize}
        \item[($1 \implies 2$)] Trivial proof
        \item[($2 \implies 3$)] We construct an $s - t$ cut. Let $S$ be the set of all vertices reachable from $s$ in $G_f$. Let $T = \v \setminus S$. Then $s \in S$ and $t \in T$ because of our assumption. By Lemma \ref{lemma:3.13_1}
        \begin{align*}
            Value(f) &= \sum_{e \in \delta^-_{G_f}(v)}f(e) - \sum_{e \in \delta^+_{G_f}(v)}f(e)\\
                     &= \sum_{e \in \delta^+_{G_f}(v)} u(e) - 0 = c(S, T)
        \end{align*}
        \item[($3 \implies 1$)] Suppose there is an $s - t$ cut $(S, T)$ with $Value(f) = c(S, T)$. Let $f'$ be a flow where $Value(f')$ is maximal:
        \begin{equation*}
            c(S, T) = Value(f) \leq Value(f') \underbrace{\leq}_{\text{By Lemma \ref{lemma:3.13}.2}} c(S, T)
        \end{equation*}
        Which means that:
        \begin{equation*}
            Value(f) = Value(f')
        \end{equation*}
        Having proven all the possible implications we have proven the theorem.\qed
    \end{itemize}
\end{prf}
\subsection{\texttt{Floyd-Fulkerson's algirithm}}
\subsubsection{Input}
    Given a network $(G, u, s, t)$
\subsubsection{Output}
    Compute the flow $f$ from $s$ to $t$ of maximum value.
\subsubsection{Algorithm}
\begin{enumerate}
    \item Set $f(e) = 0 \fa e \in \e$
    \item \texttt{while} there is a path $p$ from $s$ to $t$ in $G_f$, such that $c_f(e) > 0$ for all edges $e \in P$ \texttt{do}:
    \begin{enumerate}[label=\Alph*]
        \item Find $c_f(p) = min\{c_f(e)\st e \in p\}$
        \item \texttt{for each} $(u, v) \in p$ \texttt{do}:
        \begin{enumerate}[label=\alph*]
            \item $f(u, v) = f(u, v) + c_f(p)$ \hspace{2cm} (Send flow along the path)
            \item $f(v, u) = f(v, u) - c_f(p)$ \hspace{2cm} (The flow might be "returned" later)
        \end{enumerate}
    \end{enumerate}
\end{enumerate}
\begin{customcorollary}{3.16}
\label{corollary:3.16}
    If the capacities of a network are integers, then there exists an integral flow of max value, that is, the flow sends only integer values along the edges.
\end{customcorollary}
\begin{customtheorem}{3.17}
\label{theorem:3.17}
    Let $(G, u, s, t)$ be a network and $f$ be a flow in $G$. Then there exists a family $\mathcal{P}$ of $s - t$ paths and a family $\mathcal{C}$ of cycles in $G$ along with the weights $w: (\mathcal{P} \cup \mathcal{C}) \rightarrow \R_+$ such that.
    \begin{equation*}
        f(e) = \sum_{P \in \{\mathcal{P} \cup \mathcal{C}: e \in \e[P]\}} w(P) \hspace{2cm} \fa e \in \e
    \end{equation*}
    and
    \begin{equation*}
        Value(f) = \sum_{P \in \mathcal{P}} w(P)
    \end{equation*}
    as well as $|\mathcal{P}| + |\mathcal{C}| \leq \ne$ moreover if $f$ is integral, then $w$ can also be chosen integral.
\end{customtheorem}
\begin{prf}
    We construct $\mathcal{P}, \mathcal{C}$ and $w$ by induction on the number of edges with nonzero flow.

    If $f \equiv 0$, then the statement is trivial.

    Assume $\ex e^+ \in \e$ with $f(e^+) > 0$, consider a walk $W$ containing $e^+$ in which every edge carries positive flow and no vertex appears twice except possibly if it is one of the endpoints. Observe that $W$ contains at most $|\v|$ edges and it contains a cycle or it is a path/ If it is a path then it is an $s - t$ path and
    \begin{equation*}
        f(\delta^-(s)) = f(\delta^+(t)) = 0
    \end{equation*}
    Let $w(P) = min_{e \in \e[P]}f(e)$, set $f'(e ) = f(e) - w(P) \fa e \in \e[P]$ and $f'(e) = f(e) \fa e \in \e \setminus \e[P]$. Applying the induction hypothesis finishes the proof.\qed
\end{prf}
\begin{customcorollary}{3.18}
\label{corollary:3.18}
    Let $G$ be a (directed or undirected) graph, let $s$ and $t$ be two vertices. Then there are $k$-edge-disjoint $s - t$ paths if and only if after deleting $k - 1$ edges $t$ is still reachable from $s$.
\end{customcorollary}