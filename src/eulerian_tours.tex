\section{Eulerian tours}
\begin{definition}[Walk]
    A walk is an alternating sequence $v_1, e_1, \dots, v_{k + 1}$ of vertices and edges such that $e_i = v_i v_{i + 1}$ for all $i \in k$. 
\end{definition}
We say that the walk is simple if all the edges are distinct and we say that it's closed if $v_{k + 1} = v_1$.
\begin{definition}[Euler tour]
    An Euler tour in a graph $G$ is a simple closed walk which covers every edge of $G$.
\end{definition}
\begin{customtheorem}{Euler}
\label{theorem:euler_tour}
    A connected graph has an Euler tour $\iff$ all of its degrees are even.
\end{customtheorem}
\begin{prf}
    Before proving the theorem we will need to prove the following claims.
    \begin{claim}
    \label{claim:euler_1}
        A simple walk $w$ is closed $\iff \deg[w]{v}$ is even for all $v \in w$.
    \end{claim}
    \begin{prf}
        If $w = v_1, e_1, \dots, v_{k + 1}$ then for any $i \in \{2, \dots, k\}$ the edges $e_i$ and $e_{i + 1}$ are two distinct edges incident to $v_i$. So if $v \neq v_i, v_{k + 1}$ then $\deg[w]{v}$ is even.

        If $v_1 = v_{k + 1}$ then $e_1$ and $e_k$ are edges at $v_1$. So, if $w$ is closed then the vertices inside $w$ must have even degree, if not, $v_1$ and $v_{k + 1}$ have odd degree.\qed
    \end{prf}
    \begin{claim}
    \label{claim:euler_2}
        Any maximal simple walk in a graph whose degrees are all even is closed
    \end{claim}
    \begin{prf}
        Let $w = v_1, e_1, \dots, v_{k + 1}$, let $w$ be a maximal walk, if $w$ is not closed, then $v_{k + 1}$ has odd degree in $w$, so $\ex v_{k + 1}w \in \e \setminus \e[w]$, this means that we could extend $w$ which is absurd.\qed
    \end{prf}
    Let's now prove the theorem.

    \begin{enumerate}
        \item[($\implies$)] If $w$ is an Euler tour, then by Claim \ref{claim:euler_1} implies $\deg[w]{v}$ is even for all $v \in \v[w]$. By definition, $\deg[w]{v} = \deg{v} \fa v \in \v[w]$ and $\v[w] = \v$
        \item[($\impliedby$)] Suppose that $\deg{v}$ is even $\fa v \in \v$, let $w$ be a maximal simple walk in $G$. We want to show that $\e[w] = \e$.
        
        By Claim \ref{claim:euler_2} $w$ is closed. Say $w = v_1, e_1, \dots, e_k, v_1$. Suppose for a contradiction that there is an edge in $G' = G - \e[w]$. By Claim \ref{claim:euler_1} all degrees in $G'$ are even.

        Since $G$ is connected there is $v_i \in \v[w]$ such that $\deg[G']{v_i} \neq 0$. Let $w'$ be a maximal walk in $G'$ starting at $v_i$. By Claim \ref{claim:euler_2} $w'$ is closed. Then $v_1, e_1, \dots, e_{i - 1}, w", e_i, \dots, v_1$ is a larger walk in $G$ containing $w$, which is a contradiction to the maximality of $w$.\qed
    \end{enumerate}
\end{prf}
\subsection{\texttt{Euler's algorithm}}
\subsubsection{Input}
    A connected graph $G$ whose degrees are all even.
\subsubsection{Output}
    An Euler tour $w$.
\subsubsection{Algorithm}
The following is the main procedure
\begin{enumerate}
    \item Let $v_1$ be an arbitrary vertex and return $w$ = \texttt{Euler($v_1$)}
\end{enumerate}
The following is the \texttt{Euler($v$)} procedure
\begin{enumerate}
    \item set $w = v$ and $x = v$
    \item \label{algo:cycle_head} \texttt{if} $\deg{x} = 0$ \texttt{then} go to \ref{algo:cycle_end}
    \begin{enumerate}
        \item \texttt{else} let $y \in \ngbrs{x}$
    \end{enumerate}
    \item Set $w = w, xy, y$, set $x = y$, set $G = G - xy$ and go to \ref{algo:cycle_head}
    \item \label{algo:cycle_end} Let $v_1, e_1, \dots, e_k, v_{k + 1}$ be the sequence $w$
    \begin{enumerate}
        \item \texttt{for} $i = 2$ to $k$ \texttt{do}:
        \begin{enumerate}
            \item Set $w_i =$ \texttt{Euler($v_i$)}
        \end{enumerate}
    \end{enumerate}
    \item Set $w = v_1, e_1, w_2, e_2, \dots, w_k, e_k, w_{k + 1}$
\end{enumerate}
Before proving that the algorithm is sound let's show an example of the algorithm in motion.
\begin{figure}
    \centering
    \begin{tikzpicture}
    \node[defv] (n1) at (-3,0) {$v_1$};
    \node[defv] (n2) at (-1,3) {$v_2$};
    \node[defv] (n3) at (-1,0) {$v_3$};
    \node[defv] (n4) at (2,3) {$v_4$};
    \node[defv] (n5) at (1,1.5) {$v_5$};
    \node[defv] (n6) at (2,0) {$v_6$};
    \node[defv] (n7) at (1,-2) {$v_7$};
    \node[defv] (n8) at (-1,-2) {$v_8$};
    \node[defv] (n9) at (3.5,2) {$v_9$};
    \node[defv] (n10) at (3,-1) {$v_{10}$};

    \draw[defe] (n1) edge node[above] {} (n2);
    \draw[defe] (n1) edge node[above] {} (n3);
    \draw[defe, out=-90, in=-60, looseness=1.5] (n1) edge node[above] {} (n10);
    \draw[defe] (n2) edge node[above] {} (n4);
    \draw[defe] (n3) edge node[above] {} (n5);
    \draw[defe] (n3) edge node[above] {} (n8);
    \draw[defe] (n4) edge node[above] {} (n5);
    \draw[defe] (n5) edge node[above] {} (n6);
    \draw[defe] (n5) edge node[above] {} (n7);
    \draw[defe] (n6) edge node[above] {} (n9);
    \draw[defe] (n6) edge node[above] {} (n10);
    \draw[defe] (n8) edge node[above] {} (n7);
    \draw[defe] (n9) edge node[above] {} (n10);

    \draw[ultra thick, yellow, dashed] (n1) edge node[above] {} (n3);
    \draw[ultra thick, yellow, dashed] (n3) edge node[above] {} (n2);
    \draw[ultra thick, yellow, dashed] (n2) edge node[above] {} (n1);
    \draw[ultra thick, yellow, dashed, out=-90, in=-60, looseness=1.5] (n1) edge node[above] {} (n10);

    \draw[ultra thick, blue, dashed] (n3) edge node[above] {} (n5);
    \draw[ultra thick, blue, dashed] (n5) edge node[above] {} (n4);
    \draw[ultra thick, blue, dashed] (n2) edge node[above] {} (n4);
    \draw[ultra thick, blue, dashed] (n5) edge node[above] {} (n6);
    \draw[ultra thick, blue, dashed] (n5) edge node[above] {} (n7);
    \draw[ultra thick, blue, dashed] (n7) edge node[above] {} (n8);
    \draw[ultra thick, blue, dashed] (n8) edge node[above] {} (n3);

    \draw[ultra thick, green, dashed] (n6) edge node[above] {} (n10);
    \draw[ultra thick, green, dashed] (n10) edge node[above] {} (n9);
    \draw[ultra thick, green, dashed] (n9) edge node[above] {} (n6);


\end{tikzpicture}
    \caption{Euler's algorithm in action}
    \label{fig:eulerian_tour}
\end{figure}
With every time we execute the procedure that looks for another piece of the Eulerian cycle we are identifying one of the coloured partitions showed in Figure \ref{fig:eulerian_tour}.
\begin{customtheorem}{1.29}
\label{theorem:1.29}
    Euler's algorithm works properly and runs in $\O(m)$
\end{customtheorem}
\begin{prf}
    \begin{itemize}
        To prove the theorem we will consider Runtime and Correctness as two different concepts.
        \item [\textbf{Running time}] Across all iterations a vertex $v$ is encountered at most $\deg{v}$ times (because we delete each edge after it's explored). Note that we can run \texttt{Euler($\smth$)} at most $m$ times, thus we get a $\O(m)$ runtime.
        \item [\textbf{Correctness}] We can prove this by induction on the number of edges $\ne$. If $\ne = 0$ then $G = v_1$ and $w = v_1$. Let's now suppose we are in the case $\ne \geq 1$.
        
        By the degree condition we have $v_{k + 1} = v_1$ when we first reach \ref{algo:cycle_end}. Let $G'$ be the graph at this stage, then $w = v_1, e_1, \dots, e_k, v_1$ and $G' = G - \e[w]$. Any edge $e \in \e[G']$ lies in a component containing $v_i$ for some $i \in \{2, \dots, k\}$ thus $e$ will be contained in $w_i$ by the induction hypothesis since in \ref{algo:cycle_end} we run \texttt{Euler($v_i$)} in the component containing $v_i$.\qed
    \end{itemize}
\end{prf}